\documentclass[11pt,a4paper]{article}

% Packages
\usepackage[utf8]{inputenc}
\usepackage[T1]{fontenc}
\usepackage[spanish]{babel}
\usepackage{geometry}
\usepackage{booktabs}
\usepackage{array}
\usepackage{longtable}
\usepackage{graphicx}
\usepackage{float}
\usepackage{hyperref}
\usepackage{xcolor}
\usepackage{titlesec}
\usepackage{parskip}
\usepackage{fancyhdr}
\usepackage{palatino} % cool font
\usepackage{mathpazo} % fancy font for math

% Page geometry
\geometry{margin=2.5cm}

% Hyperref setup
\hypersetup{
    colorlinks=true,
    linkcolor=blue!60!black,
    urlcolor=blue!60!black,
    citecolor=blue!60!black
}

% Header/Footer
\pagestyle{fancy}
\fancyhf{}
\rhead{Comparación CASEN 2017-2024}
\lhead{Ingresos, Pobreza y Desigualdad en Chile}
\rfoot{Página \thepage}

% Title formatting
\titleformat{\section}{\Large\bfseries}{\thesection}{1em}{}
\titleformat{\subsection}{\large\bfseries}{\thesubsection}{1em}{}

% Custom column type for tables
\newcolumntype{R}[1]{>{\raggedleft\arraybackslash}p{#1}}
\newcolumntype{L}[1]{>{\raggedright\arraybackslash}p{#1}}
\newcolumntype{C}[1]{>{\centering\arraybackslash}p{#1}}

\begin{document}

% Title Page
\begin{titlepage}
    \centering
    \vspace*{2cm}

    {\Huge\bfseries Ingresos, Pobreza y Desigualdad en Chile}\\[0.5cm]
    {\LARGE Una Comparación de CASEN 2017 y 2024}\\[2cm]

    \vfill

    {\large Fuente de Datos: Encuesta CASEN}\\[0.3cm]
    {\large Ministerio de Desarrollo Social y Familia, Chile}\\[1cm]

    {\large Enero 2026}

    \vfill
\end{titlepage}

% Table of Contents
\tableofcontents
\newpage

% ============================================================
\section{Resumen Ejecutivo}
% ============================================================

Este informe analiza los cambios en ingresos, pobreza y desigualdad en Chile entre 2017 y 2024, utilizando datos de las encuestas CASEN (Caracterización Socioeconómica Nacional). El análisis cubre tres dimensiones principales: reducción de la pobreza y efectividad de las transferencias estatales, crecimiento de ingresos a lo largo de la distribución, y retornos laborales a la educación.

\textbf{Principales Hallazgos}:

\begin{itemize}
    \item \textbf{Reducción de la pobreza}: La pobreza por ingresos totales cayó de \textbf{9,8\% a 4,6\%}, con la eficiencia de las transferencias estatales casi duplicándose (de 27,8\% a 45,4\% de la pobreza autónoma eliminada). La pobreza en hogares con solo adultos mayores está virtualmente eliminada (0,12\%).

    \item \textbf{Crecimiento pro-pobre de ingresos}: El crecimiento de ingresos fue fuertemente progresivo---el decil más bajo creció 26\% mientras el decil más alto creció solo 7\% (ajustado por inflación). Este patrón se mantuvo tanto para ingresos totales como autónomos, indicando una compresión impulsada por el mercado.

    \item \textbf{Disminución de la prima educacional}: La prima salarial universitaria cayó sustancialmente, de 125\% a 99\% en la mediana. Los salarios por hora de trabajadores sin universidad crecieron 21\% mientras los graduados universitarios vieron solo un 7\% de crecimiento. Esto representa un estrechamiento significativo de la desigualdad basada en educación.
\end{itemize}

En conjunto, estos hallazgos sugieren que Chile experimentó una compresión generalizada de ingresos entre 2017 y 2024, con ganancias concentradas en hogares de menores ingresos y trabajadores menos educados.

% ============================================================
\section{Metodología}
% ============================================================

\subsection{Definiciones de Ingreso}

\begin{itemize}
    \item \textbf{Ingreso total}: \texttt{ytotcorh / numper} (ingreso del hogar corregido per cápita, incluyendo todas las fuentes)
    \item \textbf{Ingreso autónomo}: \texttt{(ytotcorh - ysubh) / numper} (excluyendo transferencias estatales)
    \item \textbf{Transferencias estatales (ysubh)}: Pensiones no contributivas + subsidios
    \item \textbf{Pensiones contributivas (AFP)}: Permanecen en el ingreso autónomo (autofinanciadas)
\end{itemize}

\subsection{Líneas de Pobreza}

\begin{itemize}
    \item \textbf{2017}: \$107.347 per cápita mensual (precios 2017)
    \item \textbf{2024}: \$152.160 per cápita mensual (precios 2024)
\end{itemize}

Ambas líneas de pobreza son equivalentes al estándar del Banco Mundial de \$8,3 PPP por día, ajustado a precios chilenos en cada año.

\subsection{Categorías de Hogares}

\begin{table}[H]
\centering
\begin{tabular}{@{}ll@{}}
\toprule
\textbf{Categoría} & \textbf{Definición} \\
\midrule
Sin65 & Hogares SIN miembros de 65+ años \\
AlMenos1\_65 & Hogares con al menos un miembro de 65+ años \\
Solo65 & Hogares donde TODOS los miembros tienen 65+ años \\
EdadLaboral & Hogares con solo miembros de 24--64 años \\
\bottomrule
\end{tabular}
\caption{Categorías de composición etaria de hogares}
\end{table}

% ============================================================
\section{Principales Hallazgos}
% ============================================================

\subsection{Reducción Dramática de la Pobreza (2017--2024)}

\begin{table}[H]
\centering
\begin{tabular}{@{}lrrr@{}}
\toprule
\textbf{Indicador} & \textbf{2017} & \textbf{2024} & \textbf{Cambio} \\
\midrule
Pobreza por ingreso total & 9,80\% & 4,59\% & \textbf{--5,21 pp} \\
Pobreza por ingreso autónomo & 13,57\% & 8,41\% & \textbf{--5,16 pp} \\
Impacto de transferencias & 3,77 pp & 3,82 pp & +0,05 pp \\
\bottomrule
\end{tabular}
\caption{Comparación de tasas de pobreza nacional}
\end{table}

Tanto la pobreza total como la autónoma disminuyeron en magnitudes similares ($\sim$5 pp), sugiriendo que la reducción de la pobreza fue impulsada principalmente por mejoras en los ingresos de mercado (salarios, trabajo independiente, pensiones contributivas) más que por expansión de transferencias estatales.

\subsection{La Eficiencia de las Transferencias Estatales se Duplicó}

\begin{table}[H]
\centering
\begin{tabular}{@{}lrr@{}}
\toprule
\textbf{Indicador} & \textbf{2017} & \textbf{2024} \\
\midrule
Impacto de transferencias (pp) & 3,77 & 3,82 \\
\% de pobreza autónoma eliminada & \textbf{27,8\%} & \textbf{45,4\%} \\
\bottomrule
\end{tabular}
\caption{Comparación de eficiencia de transferencias}
\end{table}

Aunque el impacto absoluto de las transferencias permaneció casi constante (3,77 pp vs 3,82 pp), la eficiencia aumentó dramáticamente porque la pobreza autónoma disminuyó. Las transferencias estatales ahora eliminan \textbf{45,4\%} de la pobreza pre-transferencias, comparado con solo 27,8\% en 2017.

\subsection{Casi Eliminación de la Pobreza Entre Adultos Mayores}

El hallazgo más notable concierne a \textbf{hogares con solo adultos mayores} (Solo65):

\begin{table}[H]
\centering
\begin{tabular}{@{}lrrr@{}}
\toprule
\textbf{Indicador} & \textbf{2017} & \textbf{2024} & \textbf{Cambio} \\
\midrule
Pobreza por ingreso total & 0,85\% & \textbf{0,12\%} & --0,73 pp \\
Pobreza por ingreso autónomo & 10,45\% & 9,34\% & --1,11 pp \\
Impacto de transferencias & 9,60 pp & 9,22 pp & --0,38 pp \\
\textbf{Eficiencia de transferencias} & 91,9\% & \textbf{98,7\%} & +6,8 pp \\
\bottomrule
\end{tabular}
\caption{Indicadores de pobreza en hogares con solo adultos mayores}
\end{table}

\textbf{Interpretación}:
\begin{itemize}
    \item Sin transferencias estatales, $\sim$9--10\% de los hogares con solo adultos mayores serían pobres (similar en ambos años)
    \item Las transferencias estatales ahora eliminan \textbf{98,7\%} de la pobreza autónoma entre hogares con solo adultos mayores
    \item La pobreza por ingreso total en hogares con solo adultos mayores ha sido virtualmente eliminada (0,12\%)
    \item Esto representa uno de los programas de transferencias focalizadas más efectivos en América Latina
\end{itemize}

\subsection{Reducción de Pobreza en Todos los Tipos de Hogares}

\begin{table}[H]
\centering
\begin{tabular}{@{}lrrr@{}}
\toprule
\textbf{Tipo de Hogar} & \textbf{2017} & \textbf{2024} & \textbf{Cambio} \\
\midrule
Sin adultos mayores (Sin65) & 11,49\% & 5,71\% & --5,78 pp \\
Al menos 1 adulto mayor & 5,66\% & 1,73\% & --3,93 pp \\
Solo adultos mayores & 0,85\% & 0,12\% & --0,73 pp \\
Solo edad laboral & 2,78\% & 1,59\% & --1,19 pp \\
\bottomrule
\end{tabular}
\caption{Pobreza por ingreso total según tipo de hogar}
\end{table}

Todos los tipos de hogares experimentaron reducción de pobreza, con hogares sin adultos mayores mostrando la mayor disminución absoluta (--5,78 pp).

\subsection{Eficiencia de Transferencias por Tipo de Hogar}

\begin{table}[H]
\centering
\begin{tabular}{@{}lrrr@{}}
\toprule
\textbf{Tipo de Hogar} & \textbf{2017} & \textbf{2024} & \textbf{Cambio} \\
\midrule
General & 27,8\% & 45,4\% & +17,6 pp \\
Sin adultos mayores & 17,6\% & 26,0\% & +8,4 pp \\
Al menos 1 adulto mayor & 55,3\% & 82,9\% & +27,6 pp \\
\textbf{Solo adultos mayores} & 91,9\% & \textbf{98,7\%} & +6,8 pp \\
Solo edad laboral & 28,5\% & 31,5\% & +3,0 pp \\
\bottomrule
\end{tabular}
\caption{Eficiencia de transferencias (\% de pobreza autónoma eliminada) por tipo de hogar}
\end{table}

\textbf{Hallazgo clave}: Los hogares con adultos mayores se benefician más de las transferencias estatales. Para hogares con al menos un adulto mayor, las transferencias ahora eliminan 82,9\% de la pobreza autónoma, aumentando desde 55,3\% en 2017.

% ============================================================
\section{Análisis Regional}
% ============================================================

\subsection{Regiones con Mayor Reducción de Pobreza}

\begin{table}[H]
\centering
\begin{tabular}{@{}lrrr@{}}
\toprule
\textbf{Región} & \textbf{2017} & \textbf{2024} & \textbf{Cambio} \\
\midrule
La Araucanía & 19,08\% & 9,30\% & \textbf{--9,78 pp} \\
Los Lagos & 12,58\% & 3,60\% & \textbf{--8,98 pp} \\
Ñuble & 15,84\% & 7,10\% & \textbf{--8,74 pp} \\
Coquimbo & 14,89\% & 6,47\% & --8,42 pp \\
Biobío & 13,31\% & 4,99\% & --8,32 pp \\
\bottomrule
\end{tabular}
\caption{Regiones con mayor reducción de pobreza}
\end{table}

Regiones históricamente pobres en el sur de Chile (La Araucanía, Los Lagos) y centro-sur (Ñuble, Maule) mostraron las mayores mejoras.

\subsection{Única Región con Aumento de Pobreza}

\begin{table}[H]
\centering
\begin{tabular}{@{}lrrr@{}}
\toprule
\textbf{Región} & \textbf{2017} & \textbf{2024} & \textbf{Cambio} \\
\midrule
Magallanes & 2,25\% & 2,43\% & +0,18 pp \\
\bottomrule
\end{tabular}
\caption{Región con aumento de pobreza}
\end{table}

Magallanes fue la única región donde la pobreza aumentó levemente, aunque permanece entre las regiones con menor pobreza.

\subsection{Regiones con Mayor Aumento en Efectividad de Transferencias}

\begin{table}[H]
\centering
\begin{tabular}{@{}lrrr@{}}
\toprule
\textbf{Región} & \textbf{Impacto 2017} & \textbf{Impacto 2024} & \textbf{Cambio} \\
\midrule
Atacama & 1,81 pp & 3,09 pp & +1,28 pp \\
Biobío & 5,03 pp & 5,66 pp & +0,63 pp \\
Valparaíso & 3,57 pp & 3,95 pp & +0,38 pp \\
\bottomrule
\end{tabular}
\caption{Regiones con mayor aumento en efectividad de transferencias}
\end{table}

% ============================================================
\section{Cambios Demográficos}
% ============================================================

\subsection{Cambios en Composición de Hogares (2017--2024)}

\begin{table}[H]
\centering
\begin{tabular}{@{}lrrr@{}}
\toprule
\textbf{Categoría} & \textbf{2017} & \textbf{2024} & \textbf{Cambio} \\
\midrule
Sin adultos mayores & 71,09\% & 71,84\% & +0,75 pp \\
Al menos 1 adulto mayor & 28,91\% & 28,16\% & --0,75 pp \\
Solo adultos mayores & 5,59\% & 5,77\% & +0,18 pp \\
\textbf{Solo edad laboral} & 13,64\% & \textbf{17,92\%} & \textbf{+4,28 pp} \\
\bottomrule
\end{tabular}
\caption{Participación poblacional por composición del hogar}
\end{table}

La proporción de hogares con solo personas en edad laboral (24--64, sin niños ni adultos mayores) aumentó en 4,28 puntos porcentuales, reflejando cambios demográficos y posiblemente postergación de la maternidad/paternidad.

% ============================================================
\section{Crecimiento de Ingresos por Decil}
% ============================================================

Esta sección analiza cómo el ingreso promedio per cápita cambió a través de la distribución de ingresos entre 2017 y 2024. Todos los valores de 2017 están ajustados por inflación (factor 1,42) para expresarlos en pesos chilenos de 2024.

\subsection{Crecimiento del Ingreso Total por Decil}

\begin{table}[H]
\centering
\begin{tabular}{@{}crrr@{}}
\toprule
\textbf{Decil} & \textbf{2017 (adj.)} & \textbf{2024} & \textbf{Cambio} \\
\midrule
1 (más pobre) & 111.534 & 141.011 & \textbf{+26,4\%} \\
2 & 172.535 & 221.021 & \textbf{+28,1\%} \\
3 & 215.890 & 275.883 & \textbf{+27,8\%} \\
4 & 258.262 & 330.271 & +27,9\% \\
5 & 304.404 & 386.745 & +27,0\% \\
6 & 359.338 & 451.587 & +25,7\% \\
7 & 431.036 & 532.548 & +23,6\% \\
8 & 535.356 & 649.377 & +21,3\% \\
9 & 732.837 & 862.796 & +17,7\% \\
10 (más rico) & 1.813.019 & 1.943.776 & +7,2\% \\
\bottomrule
\end{tabular}
\caption{Ingreso total promedio per cápita por decil (CLP, precios 2024)}
\end{table}

\textbf{Hallazgo clave}: El crecimiento de ingresos fue \textbf{fuertemente pro-pobre}. Los deciles inferiores experimentaron ganancias de 26--28\%, mientras el decil superior creció solo 7,2\%. Esto representa una compresión significativa de la distribución de ingresos.

\subsection{Crecimiento del Ingreso Autónomo por Decil}

\begin{table}[H]
\centering
\begin{tabular}{@{}crrr@{}}
\toprule
\textbf{Decil} & \textbf{2017 (adj.)} & \textbf{2024} & \textbf{Cambio} \\
\midrule
1 (más pobre) & 97.896 & 122.384 & \textbf{+25,0\%} \\
2 & 156.345 & 195.214 & \textbf{+24,9\%} \\
3 & 197.737 & 248.600 & \textbf{+25,7\%} \\
4 & 240.487 & 295.924 & +23,1\% \\
5 & 285.076 & 348.054 & +22,1\% \\
6 & 341.171 & 408.862 & +19,8\% \\
7 & 415.935 & 492.178 & +18,3\% \\
8 & 522.811 & 610.859 & +16,8\% \\
9 & 722.737 & 833.368 & +15,3\% \\
10 (más rico) & 1.807.877 & 1.931.629 & +6,8\% \\
\bottomrule
\end{tabular}
\caption{Ingreso autónomo promedio per cápita por decil (CLP, precios 2024)}
\end{table}

\textbf{Hallazgo clave}: El ingreso autónomo (excluyendo transferencias estatales) también mostró crecimiento pro-pobre, aunque ligeramente menos pronunciado que el ingreso total. Esto indica que las mejoras en ingresos de mercado---salarios, empleo, pensiones contributivas---beneficiaron a los deciles inferiores más que a los superiores.

\subsection{Contribución de Transferencias por Decil}

La diferencia entre el crecimiento del ingreso total y autónomo revela el rol de las transferencias estatales:

\begin{table}[H]
\centering
\begin{tabular}{@{}ccc@{}}
\toprule
\textbf{Decil} & \textbf{Crec. Total} & \textbf{Crec. Autónomo} \\
\midrule
1 & +26,4\% & +25,0\% \\
2 & +28,1\% & +24,9\% \\
3 & +27,8\% & +25,7\% \\
10 & +7,2\% & +6,8\% \\
\bottomrule
\end{tabular}
\caption{Comparación de crecimiento de ingreso total vs autónomo (deciles seleccionados)}
\end{table}

La pequeña brecha entre el crecimiento total y autónomo (1--3 puntos porcentuales en deciles inferiores) sugiere que aunque las transferencias contribuyen a la reducción de pobreza, la mayor parte del crecimiento de ingresos provino de fuentes de mercado.

% ============================================================
\section{Análisis de Distribución de Ingresos}
% ============================================================

\subsection{Hogares con Solo Adultos Mayores por Decil de Ingreso}

Usando los límites de deciles de ingreso total, los hogares con solo adultos mayores se desplazaron hacia deciles más altos:

\begin{table}[H]
\centering
\begin{tabular}{@{}crrrr@{}}
\toprule
\textbf{Decil} & \textbf{Total 2017} & \textbf{Total 2024} & \textbf{Autón. 2017} & \textbf{Autón. 2024} \\
\midrule
1 (más pobre) & 0,85\% & 0,22\% & 10,51\% & 13,86\% \\
2 & 2,08\% & 1,73\% & 5,39\% & 11,66\% \\
3 & 4,93\% & 2,53\% & 7,05\% & 9,41\% \\
\midrule
\textbf{3 inferiores total} & \textbf{7,86\%} & \textbf{4,48\%} & \textbf{22,95\%} & \textbf{34,93\%} \\
\bottomrule
\end{tabular}
\caption{Distribución de hogares con solo adultos mayores por decil de ingreso}
\end{table}

\textbf{Observación crítica}:
\begin{itemize}
    \item Cuando se mide por \textbf{ingreso total}, los hogares con solo adultos mayores subieron en la distribución (menos en deciles inferiores)
    \item Cuando se mide por \textbf{ingreso autónomo}, los hogares con solo adultos mayores bajaron (más en deciles inferiores)
\end{itemize}

Esta divergencia revela que las transferencias estatales están haciendo más trabajo para elevar a los hogares de adultos mayores fuera de la pobreza. Sin transferencias, 34,9\% de los hogares con solo adultos mayores estarían en los tres deciles inferiores (aumentando desde 23\% en 2017). Las transferencias los mueven a deciles medios y superiores.

% ============================================================
\section{Ingreso Laboral y Prima Educacional}
% ============================================================

Esta sección analiza el ingreso laboral por hora para trabajadores empleados (dependientes de un empleador, edades 26--65) por cohorte de edad y nivel educacional. Todos los valores de 2017 están ajustados por inflación a pesos chilenos de 2024.

\subsection{Ingreso Laboral por Hora según Cohorte de Edad y Educación}

\begin{table}[H]
\centering
\small
\begin{tabular}{@{}lrrrrrr@{}}
\toprule
& \multicolumn{3}{c}{\textbf{2017}} & \multicolumn{3}{c}{\textbf{2024}} \\
\cmidrule(lr){2-4} \cmidrule(lr){5-7}
\textbf{Cohorte} & Total & Sin Univ & Univ & Total & Sin Univ & Univ \\
\midrule
26-30 & 729 & 586 & 1.002 & 866 & 722 & 1.050 \\
31-35 & 820 & 621 & 1.230 & 1.010 & 722 & 1.283 \\
36-40 & 807 & 615 & 1.342 & 1.050 & 722 & 1.443 \\
41-46 & 729 & 615 & 1.366 & 1.050 & 722 & 1.575 \\
46-50 & 729 & 598 & 1.366 & 945 & 722 & 1.575 \\
51-55 & 697 & 583 & 1.458 & 866 & 722 & 1.443 \\
56-60 & 685 & 574 & 1.458 & 825 & 722 & 1.540 \\
61-65 & 692 & 583 & 1.528 & 787 & 722 & 1.575 \\
\midrule
\textbf{Promedio} & 736 & 597 & 1.344 & 925 & 722 & 1.436 \\
\bottomrule
\end{tabular}
\caption{Ingreso laboral mediano por hora según cohorte de edad y educación (CLP/hora, precios 2024)}
\label{tab:income_median}
\end{table}

\textbf{Observaciones clave}:
\begin{itemize}
    \item El ingreso mediano por hora para trabajadores \textbf{sin educación universitaria} aumentó de 597 a 722 CLP/hora (+21\%)
    \item El ingreso mediano por hora para trabajadores \textbf{con educación universitaria} aumentó de 1.344 a 1.436 CLP/hora (+7\%)
    \item Los trabajadores sin universidad vieron mayores ganancias proporcionales, estrechando la brecha educacional
\end{itemize}

\subsection{Prima Educacional Universitaria}

La prima educacional mide cuánto más ganan los trabajadores con educación universitaria comparados con aquellos sin educación universitaria.

\begin{table}[H]
\centering
\begin{tabular}{@{}lrrr@{}}
\toprule
\textbf{Cohorte} & \textbf{2017} & \textbf{2024} & \textbf{Cambio} \\
\midrule
26-30 & 71,0\% & 45,4\% & --25,6 pp \\
31-35 & 98,1\% & 77,7\% & --20,4 pp \\
36-40 & 118,2\% & 99,9\% & --18,3 pp \\
41-46 & 122,1\% & 118,1\% & --4,0 pp \\
46-50 & 128,4\% & 118,1\% & --10,3 pp \\
51-55 & 150,1\% & 99,9\% & --50,2 pp \\
56-60 & 154,0\% & 113,3\% & --40,7 pp \\
61-65 & 162,1\% & 118,1\% & --44,0 pp \\
\midrule
\textbf{Promedio} & 125,5\% & 98,8\% & \textbf{--26,7 pp} \\
\bottomrule
\end{tabular}
\caption{Prima educacional universitaria sobre ingreso mediano por hora}
\label{tab:premium_median}
\end{table}

\textbf{Hallazgo notable}: La prima educacional universitaria \textbf{disminuyó sustancialmente} en todas las cohortes de edad. En promedio, la prima mediana cayó de 125,5\% a 98,8\% (una caída de 26,7 puntos porcentuales). Esto significa:
\begin{itemize}
    \item En 2017, los graduados universitarios ganaban 2,25$\times$ el salario por hora de los no graduados
    \item En 2024, los graduados universitarios ganan 1,99$\times$ el salario por hora de los no graduados
\end{itemize}

\subsection{Prima Educacional a través de la Distribución de Ingresos}

Las siguientes figuras muestran cómo la prima educacional cambió en diferentes puntos de la distribución de ingresos por hora.

\begin{figure}[H]
\centering
\includegraphics[width=0.85\textwidth]{education_premium_p25_es.png}
\caption{Prima educacional universitaria en P25 (trabajadores de menores ingresos)}
\label{fig:premium_p25}
\end{figure}

\begin{figure}[H]
\centering
\includegraphics[width=0.85\textwidth]{education_premium_median_es.png}
\caption{Prima educacional universitaria en la mediana}
\label{fig:premium_median}
\end{figure}

\begin{figure}[H]
\centering
\includegraphics[width=0.85\textwidth]{education_premium_p75_es.png}
\caption{Prima educacional universitaria en P75 (trabajadores de mayores ingresos)}
\label{fig:premium_p75}
\end{figure}

\begin{figure}[H]
\centering
\includegraphics[width=0.85\textwidth]{education_premium_p90_es.png}
\caption{Prima educacional universitaria en P90 (trabajadores de ingresos más altos)}
\label{fig:premium_p90}
\end{figure}

\subsection{Resumen: Disminución de Retornos a la Educación Universitaria}

\begin{table}[H]
\centering
\begin{tabular}{@{}lrrr@{}}
\toprule
\textbf{Percentil} & \textbf{Prima 2017} & \textbf{Prima 2024} & \textbf{Cambio} \\
\midrule
P25 & 61,7\% & 40,8\% & --20,9 pp \\
Mediana & 125,5\% & 98,8\% & --26,7 pp \\
P75 & 179,6\% & 155,4\% & --24,2 pp \\
P90 & 226,4\% & 211,8\% & --14,6 pp \\
\bottomrule
\end{tabular}
\caption{Prima educacional promedio por percentil (todas las cohortes)}
\label{tab:premium_summary}
\end{table}

La prima educacional disminuyó en todos los puntos de la distribución, con las mayores caídas en la mediana y P75. Esto sugiere:
\begin{itemize}
    \item Los salarios de trabajadores sin universidad crecieron más rápido que los de graduados universitarios
    \item El mercado laboral puede estar experimentando compresión en la mitad y mitad superior de la distribución
    \item Los aumentos del salario mínimo y el ajuste del mercado laboral pueden haber beneficiado desproporcionadamente a trabajadores de menor calificación
\end{itemize}

% ============================================================
\section{Conclusiones Principales}
% ============================================================

\subsection{El Sistema de Transferencias de Chile para Adultos Mayores es Altamente Efectivo}

Las transferencias estatales eliminan \textbf{98,7\%} de la pobreza autónoma entre hogares con solo adultos mayores, reduciendo la pobreza por ingreso total a solo 0,12\%. Esta es una política focalizada notablemente exitosa.

\subsection{La Reducción General de Pobreza fue Impulsada por Mejoras en Ingresos de Mercado}

La disminución paralela en pobreza total y autónoma (--5,2 pp cada una) indica que el crecimiento económico, el empleo y los salarios de mercado mejoraron sustancialmente entre 2017 y 2024.

\subsection{La Eficiencia de las Transferencias Casi se Duplicó}

Aunque el impacto absoluto de las transferencias en reducción de pobreza permaneció constante ($\sim$3,8 pp), las transferencias ahora eliminan una proporción mucho mayor de la pobreza autónoma (45\% vs 28\%), ya que la base de pobres autónomos disminuyó.

\subsection{Convergencia Regional}

Las regiones históricamente pobres (La Araucanía, Ñuble, Los Lagos) mostraron las mayores reducciones de pobreza, sugiriendo cierta convergencia regional en niveles de vida.

\subsection{Creciente Dependencia de los Adultos Mayores en Transferencias}

Sin transferencias, más hogares de adultos mayores caerían en pobreza en 2024 que en 2017 (medido por distribución de ingreso autónomo). El sistema de transferencias está compensando lo que de otra manera sería un deterioro de la posición de hogares de adultos mayores en la distribución de ingresos.

\subsection{Disminución de Retornos a la Educación Universitaria}

La prima salarial universitaria cayó de 125\% a 99\% en la mediana, impulsada por un crecimiento salarial más rápido para trabajadores sin universidad (+21\%) comparado con graduados universitarios (+7\%). Esta compresión de la prima educacional ocurrió en todas las cohortes de edad y en todos los puntos de la distribución salarial, sugiriendo cambios estructurales en el mercado laboral que favorecen a trabajadores menos educados.

% ============================================================
\section{Fuentes de Datos y Archivos}
% ============================================================

\begin{itemize}
    \item \textbf{Datos Crudos}: \texttt{raw\_data/CASEN\_2017.dta}, \texttt{raw\_data/casen\_2024.dta}
    \item \textbf{Datos Armonizados}: \texttt{harmonized\_data/casen\_subset\_2017.dta}, \texttt{harmonized\_data/casen\_subset\_2024.dta}
    \item \textbf{Análisis de Pobreza}: \texttt{output/poverty\_comparison\_extended\_2017.xlsx}, \texttt{output/poverty\_comparison\_extended\_2024.xlsx}
    \item \textbf{Análisis de Ingreso Laboral}: \texttt{output/labor\_income\_by\_cohort.xlsx}, \texttt{output/labor\_income\_tables.xlsx}
    \item \textbf{Código de Análisis}: \texttt{codes/poverty\_analysis.R}, \texttt{codes/compare\_2017\_2024.R}, \texttt{codes/labor\_income\_by\_cohort.R}
\end{itemize}

% ============================================================
\section{Notas Técnicas}
% ============================================================

\begin{enumerate}
    \item Todas las tasas de pobreza están ponderadas usando factores de expansión de la encuesta (\texttt{expr})
    \item Las líneas de pobreza se basan en el estándar de \$8,3 PPP por día del Banco Mundial (pobreza moderada)
    \item Metodología consistente en ambos años: \texttt{ytotcorh/numper} para ingreso total, \texttt{(ytotcorh-ysubh)/numper} para ingreso autónomo
    \item El análisis regional cubre las 16 regiones chilenas
    \item El análisis comunal está limitado a la Región Metropolitana
\end{enumerate}

\vfill

\begin{center}
\rule{0.5\textwidth}{0.4pt}\\[0.5cm]
\textit{Informe generado: Enero 2026}\\
\textit{Datos: CASEN 2017 y CASEN 2024, Ministerio de Desarrollo Social y Familia, Chile}
\end{center}

\end{document}
