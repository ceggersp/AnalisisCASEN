\documentclass[11pt,a4paper]{article}

% Packages
\usepackage[utf8]{inputenc}
\usepackage[T1]{fontenc}
\usepackage[spanish]{babel}
\usepackage{geometry}
\usepackage{booktabs}
\usepackage{array}
\usepackage{longtable}
\usepackage{graphicx}
\usepackage{float}
\usepackage{hyperref}
\usepackage{xcolor}
\usepackage{titlesec}
\usepackage{parskip}
\usepackage{fancyhdr}
\usepackage{palatino} % cool font
\usepackage{mathpazo} % fancy font for math

% Page geometry
\geometry{margin=2.5cm}

% Hyperref setup
\hypersetup{
    colorlinks=true,
    linkcolor=blue!60!black,
    urlcolor=blue!60!black,
    citecolor=blue!60!black
}

% Header/Footer
\pagestyle{fancy}
\fancyhf{}
\rhead{Comparación CASEN 2017-2024}
\lhead{Ingresos, Pobreza y Desigualdad en Chile}
\rfoot{Página \thepage}

% Title formatting
\titleformat{\section}{\Large\bfseries}{\thesection}{1em}{}
\titleformat{\subsection}{\large\bfseries}{\thesubsection}{1em}{}

% Custom column type for tables
\newcolumntype{R}[1]{>{\raggedleft\arraybackslash}p{#1}}
\newcolumntype{L}[1]{>{\raggedright\arraybackslash}p{#1}}
\newcolumntype{C}[1]{>{\centering\arraybackslash}p{#1}}

\begin{document}

% Title Page
\begin{titlepage}
    \centering
    \vspace*{2cm}

    {\Huge\bfseries Ingresos, Pobreza y Desigualdad en Chile}\\[0.5cm]
    {\LARGE Una Comparación de CASEN 2017 y 2024}\\[2cm]

    \vfill

    {\large Fuente de Datos: Encuesta CASEN}\\[0.3cm]
    {\large Ministerio de Desarrollo Social y Familia, Chile}\\[1cm]

    {\large Enero 2026}

    \vfill
\end{titlepage}

% Table of Contents
\tableofcontents
\newpage

% ============================================================
\section{Resumen Ejecutivo}
% ============================================================

Este informe analiza los cambios en ingresos, pobreza y desigualdad en Chile entre 2017 y 2024, utilizando datos de las encuestas CASEN (Caracterización Socioeconómica Nacional). El análisis cubre tres dimensiones principales: reducción de la pobreza y efectividad de las transferencias estatales, crecimiento de ingresos a lo largo de la distribución, y retornos laborales a la educación.

\textbf{Principales Hallazgos}:

\begin{itemize}
    \item \textbf{Reducción de la pobreza}: La pobreza por ingresos totales cayó de \textbf{9,8\% a 4,6\%}, mientras la pobreza por ingreso autónomo cayó de 25,0\% a 19,3\%. La eficiencia de las transferencias mejoró sustancialmente (de 60,8\% a 76,2\% de la pobreza autónoma eliminada). La pobreza en hogares con solo adultos mayores está virtualmente eliminada (0,12\%).

    \item \textbf{Masivo impacto de transferencias en adultos mayores}: La pobreza autónoma entre hogares con solo adultos mayores \textbf{aumentó} de 31,7\% a 41,7\%, pero las transferencias estatales ahora eliminan \textbf{99,7\%} de esa pobreza autónoma.

    \item \textbf{Crecimiento pro-pobre de ingresos}: El crecimiento de ingresos fue fuertemente progresivo---el decil más bajo creció 51\% en ingreso total mientras el decil más alto creció solo 19\% (ajustado por inflación). El ingreso autónomo también mostró patrones pro-pobres.

    \item \textbf{Disminución de la prima educacional}: La prima salarial universitaria cayó sustancialmente, de 126\% a 99\% en la mediana. Los salarios por hora de trabajadores sin universidad crecieron 21\% mientras los graduados universitarios vieron solo un 7\% de crecimiento.
\end{itemize}

En conjunto, estos hallazgos sugieren que Chile experimentó una compresión generalizada de ingresos entre 2017 y 2024, impulsada tanto por la expansión de transferencias estatales como por mejoras en ingresos de mercado concentradas en hogares de menores ingresos.

% ============================================================
\section{Metodología}
% ============================================================

\subsection{Definiciones de Ingreso}

\begin{itemize}
    \item \textbf{Ingreso total}: \texttt{ytotcorh / numper} (ingreso del hogar corregido per cápita, incluyendo todas las fuentes)
    \item \textbf{Ingreso autónomo}: \texttt{yautcorh / numper} (incluye solo ingreso laboral, trabajo independiente y pensiones contributivas---excluye todas las transferencias estatales)
    \item \textbf{Transferencias estatales}: Pensiones no contributivas (PGU, PBS) + subsidios + otras transferencias gubernamentales
    \item \textbf{Pensiones contributivas (AFP)}: Incluidas en el ingreso autónomo (autofinanciadas)
\end{itemize}

\subsection{Líneas de Pobreza}

\begin{itemize}
    \item \textbf{2017}: \$107.347 per cápita mensual (precios 2017)
    \item \textbf{2024}: \$152.160 per cápita mensual (precios 2024)
\end{itemize}

Ambas líneas de pobreza son equivalentes al estándar del Banco Mundial de \$8,3 PPP por día, ajustado a precios chilenos en cada año.

\subsection{Categorías de Hogares}

\begin{table}[H]
\centering
\caption{Categorías de composición etaria de hogares}
\begin{tabular}{@{}ll@{}}
\toprule
\textbf{Categoría} & \textbf{Definición} \\
\midrule
Sin65 & Hogares SIN miembros de 65+ años \\
AlMenos1\_65 & Hogares con al menos un miembro de 65+ años \\
Solo65 & Hogares donde TODOS los miembros tienen 65+ años \\
EdadLaboral & Hogares con solo miembros de 24--64 años \\
\bottomrule
\end{tabular}
\end{table}

% ============================================================
\section{Principales Hallazgos}
% ============================================================

\subsection{Reducción Dramática de la Pobreza (2017--2024)}

\begin{table}[H]
\centering
\caption{Comparación de tasas de pobreza nacional}
\begin{tabular}{@{}lrrr@{}}
\toprule
\textbf{Indicador} & \textbf{2017} & \textbf{2024} & \textbf{Cambio} \\
\midrule
Pobreza por ingreso total & 9,80\% & 4,59\% & \textbf{--5,21 pp} \\
Pobreza por ingreso autónomo & 25,01\% & 19,32\% & \textbf{--5,69 pp} \\
Impacto de transferencias & 15,21 pp & 14,73 pp & --0,48 pp \\
\bottomrule
\end{tabular}
\end{table}

Las transferencias estatales tienen un impacto masivo en la reducción de pobreza, eliminando aproximadamente 15 puntos porcentuales de pobreza en ambos años. Tanto la pobreza total como la autónoma disminuyeron en magnitudes similares ($\sim$5 pp), indicando que tanto las mejoras en ingresos de mercado como la efectividad de las transferencias contribuyeron a la reducción de pobreza.

\subsection{La Eficiencia de las Transferencias Estatales Mejoró}

\begin{table}[H]
\centering
\caption{Comparación de eficiencia de transferencias}
\begin{tabular}{@{}lrr@{}}
\toprule
\textbf{Indicador} & \textbf{2017} & \textbf{2024} \\
\midrule
Impacto de transferencias (pp) & 15,21 & 14,73 \\
\% de pobreza autónoma eliminada & \textbf{60,8\%} & \textbf{76,2\%} \\
\bottomrule
\end{tabular}
\end{table}

Las transferencias estatales eliminan una proporción sustancial de la pobreza autónoma. La eficiencia mejoró de 60,8\% a 76,2\%, lo que significa que las transferencias ahora eliminan más de tres cuartos de la pobreza pre-transferencias. Esta mejora ocurrió aunque el impacto absoluto permaneció aproximadamente constante ($\sim$15 pp).

\subsection{Casi Eliminación de la Pobreza Entre Adultos Mayores}

El hallazgo más notable concierne a \textbf{hogares con solo adultos mayores} (Solo65):

\begin{table}[H]
\centering
\caption{Indicadores de pobreza en hogares con solo adultos mayores}
\begin{tabular}{@{}lrrr@{}}
\toprule
\textbf{Indicador} & \textbf{2017} & \textbf{2024} & \textbf{Cambio} \\
\midrule
Pobreza por ingreso total & 0,85\% & \textbf{0,12\%} & --0,73 pp \\
Pobreza por ingreso autónomo & 31,71\% & 41,72\% & +10,01 pp \\
Impacto de transferencias & 30,86 pp & 41,60 pp & +10,74 pp \\
\textbf{Eficiencia de transferencias} & 97,3\% & \textbf{99,7\%} & +2,4 pp \\
\bottomrule
\end{tabular}
\end{table}

\textbf{Interpretación}:
\begin{itemize}
    \item Sin transferencias estatales, la pobreza autónoma entre hogares con solo adultos mayores \textbf{aumentó} de 31,7\% a 41,7\%
    \item Sin embargo, las transferencias estatales ahora eliminan \textbf{99,7\%} de la pobreza autónoma entre hogares con solo adultos mayores
    \item La pobreza por ingreso total ha sido virtualmente eliminada (0,12\%)
    \item Esto representa uno de los programas de transferencias focalizadas más efectivos en América Latina, compensando el deterioro de la posición de ingresos autónomos
\end{itemize}

\subsection{Reducción de Pobreza en Todos los Tipos de Hogares}

\begin{table}[H]
\centering
\caption{Pobreza por ingreso total según tipo de hogar}
\begin{tabular}{@{}lrrr@{}}
\toprule
\textbf{Tipo de Hogar} & \textbf{2017} & \textbf{2024} & \textbf{Cambio} \\
\midrule
Sin adultos mayores (Sin65) & 11,49\% & 5,71\% & --5,78 pp \\
Al menos 1 adulto mayor & 5,66\% & 1,73\% & --3,93 pp \\
Solo adultos mayores & 0,85\% & 0,12\% & --0,73 pp \\
Solo edad laboral & 2,78\% & 1,59\% & --1,19 pp \\
\bottomrule
\end{tabular}
\end{table}

Todos los tipos de hogares experimentaron reducción de pobreza, con hogares sin adultos mayores mostrando la mayor disminución absoluta (--5,78 pp).

\subsection{Eficiencia de Transferencias por Tipo de Hogar}

\begin{table}[H]
\centering
\caption{Eficiencia de transferencias (\% de pobreza autónoma eliminada) por tipo de hogar}
\begin{tabular}{@{}lrrr@{}}
\toprule
\textbf{Tipo de Hogar} & \textbf{2017} & \textbf{2024} & \textbf{Cambio} \\
\midrule
General & 60,8\% & 76,2\% & +15,4 pp \\
Sin adultos mayores & 52,4\% & 64,2\% & +11,8 pp \\
Al menos 1 adulto mayor & 79,2\% & 93,8\% & +14,6 pp \\
\textbf{Solo adultos mayores} & 97,3\% & \textbf{99,7\%} & +2,4 pp \\
Solo edad laboral & 70,6\% & 79,7\% & +9,1 pp \\
\bottomrule
\end{tabular}
\end{table}

\textbf{Hallazgo clave}: La eficiencia de transferencias mejoró en todos los tipos de hogares. Para hogares con al menos un adulto mayor, las transferencias ahora eliminan 93,8\% de la pobreza autónoma, aumentando desde 79,2\% en 2017. Para hogares con solo adultos mayores, la eficiencia alcanza 99,7\%.

% ============================================================
\section{Análisis Regional}
% ============================================================

\subsection{Regiones con Mayor Reducción de Pobreza}

\begin{table}[H]
\centering
\caption{Regiones con mayor reducción de pobreza}
\begin{tabular}{@{}lrrr@{}}
\toprule
\textbf{Región} & \textbf{2017} & \textbf{2024} & \textbf{Cambio} \\
\midrule
La Araucanía & 19,08\% & 9,30\% & \textbf{--9,78 pp} \\
Los Lagos & 12,58\% & 3,60\% & \textbf{--8,98 pp} \\
Ñuble & 15,84\% & 7,10\% & \textbf{--8,74 pp} \\
Coquimbo & 14,89\% & 6,47\% & --8,42 pp \\
Biobío & 13,31\% & 4,99\% & --8,32 pp \\
\bottomrule
\end{tabular}
\end{table}

Regiones históricamente pobres en el sur de Chile (La Araucanía, Los Lagos) y centro-sur (Ñuble, Maule) mostraron las mayores mejoras.

\subsection{Única Región con Aumento de Pobreza}

\begin{table}[H]
\centering
\caption{Región con aumento de pobreza}
\begin{tabular}{@{}lrrr@{}}
\toprule
\textbf{Región} & \textbf{2017} & \textbf{2024} & \textbf{Cambio} \\
\midrule
Magallanes & 2,25\% & 2,43\% & +0,18 pp \\
\bottomrule
\end{tabular}
\end{table}

Magallanes fue la única región donde la pobreza aumentó levemente, aunque permanece entre las regiones con menor pobreza.

\subsection{Regiones con Mayor Aumento en Efectividad de Transferencias}

\begin{table}[H]
\centering
\caption{Regiones con mayor aumento en efectividad de transferencias}
\begin{tabular}{@{}lrrr@{}}
\toprule
\textbf{Región} & \textbf{Impacto 2017} & \textbf{Impacto 2024} & \textbf{Cambio} \\
\midrule
Atacama & 1,81 pp & 3,09 pp & +1,28 pp \\
Biobío & 5,03 pp & 5,66 pp & +0,63 pp \\
Valparaíso & 3,57 pp & 3,95 pp & +0,38 pp \\
\bottomrule
\end{tabular}
\end{table}

% ============================================================
\section{Cambios Demográficos}
% ============================================================

\subsection{Cambios en Composición de Hogares (2017--2024)}

\begin{table}[H]
\centering
\caption{Participación poblacional por composición del hogar}
\begin{tabular}{@{}lrrr@{}}
\toprule
\textbf{Categoría} & \textbf{2017} & \textbf{2024} & \textbf{Cambio} \\
\midrule
Sin adultos mayores & 71,09\% & 71,84\% & +0,75 pp \\
Al menos 1 adulto mayor & 28,91\% & 28,16\% & --0,75 pp \\
Solo adultos mayores & 5,59\% & 5,77\% & +0,18 pp \\
\textbf{Solo edad laboral} & 13,64\% & \textbf{17,92\%} & \textbf{+4,28 pp} \\
\bottomrule
\end{tabular}
\end{table}

La proporción de hogares con solo personas en edad laboral (24--64, sin niños ni adultos mayores) aumentó en 4,28 puntos porcentuales, reflejando cambios demográficos y posiblemente postergación de la maternidad/paternidad.

% ============================================================
\section{Crecimiento de Ingresos por Decil}
% ============================================================

Esta sección analiza cómo el ingreso promedio per cápita cambió a través de la distribución de ingresos entre 2017 y 2024. Todos los valores de 2017 están ajustados por inflación (factor 1,42) para expresarlos en pesos chilenos de 2024.

\subsection{Crecimiento del Ingreso Total por Decil}

Nota: Los deciles están definidos en base a la distribución del \textbf{ingreso autónomo}.

\begin{table}[H]
\centering
\caption{Ingreso total promedio per cápita por decil (CLP, precios 2024)}
\begin{tabular}{@{}crrr@{}}
\toprule
\textbf{Decil} & \textbf{2017 (adj.)} & \textbf{2024} & \textbf{Cambio} \\
\midrule
1 (más pobre) & 157.222 & 237.807 & \textbf{+51,3\%} \\
2 & 187.385 & 247.117 & \textbf{+31,9\%} \\
3 & 230.425 & 293.611 & \textbf{+27,4\%} \\
4 & 274.886 & 344.358 & +25,3\% \\
5 & 315.314 & 401.367 & +27,3\% \\
6 & 374.961 & 471.540 & +25,8\% \\
7 & 446.590 & 571.221 & +27,9\% \\
8 & 555.216 & 715.263 & +28,8\% \\
9 & 768.212 & 1.006.468 & +31,0\% \\
10 (más rico) & 1.899.884 & 2.256.871 & +18,8\% \\
\bottomrule
\end{tabular}
\end{table}

\textbf{Hallazgo clave}: El crecimiento de ingresos fue \textbf{fuertemente pro-pobre}. El decil inferior experimentó ganancias de 51\%, impulsado en gran parte por transferencias estatales, mientras el decil superior creció solo 19\%. Esto representa una compresión significativa de la distribución de ingresos.

\subsection{Crecimiento del Ingreso Autónomo por Decil}

\begin{table}[H]
\centering
\caption{Ingreso autónomo promedio per cápita por decil (CLP, precios 2024)}
\begin{tabular}{@{}crrr@{}}
\toprule
\textbf{Decil} & \textbf{2017 (adj.)} & \textbf{2024} & \textbf{Cambio} \\
\midrule
1 (más pobre) & 50.063 & 50.259 & \textbf{+0,4\%} \\
2 & 113.552 & 131.421 & +15,7\% \\
3 & 153.330 & 182.307 & +18,9\% \\
4 & 193.783 & 234.684 & +21,1\% \\
5 & 237.185 & 290.200 & +22,4\% \\
6 & 289.602 & 360.751 & +24,6\% \\
7 & 359.415 & 454.437 & +26,4\% \\
8 & 461.249 & 590.804 & +28,1\% \\
9 & 657.777 & 861.967 & +31,0\% \\
10 (más rico) & 1.713.050 & 2.037.644 & +18,9\% \\
\bottomrule
\end{tabular}
\end{table}

\textbf{Hallazgo clave}: El crecimiento del ingreso autónomo fue relativamente plano en la base (el decil 1 creció solo 0,4\%) pero más fuerte en los deciles medios y medio-altos. La combinación de crecimiento plano del ingreso autónomo en la base con fuerte crecimiento del ingreso total (51\%) revela el rol crucial de las transferencias estatales en elevar a los hogares más pobres.

\subsection{Contribución de Transferencias por Decil}

La diferencia entre el crecimiento del ingreso total y autónomo revela el rol crítico de las transferencias estatales:

\begin{table}[H]
\centering
\caption{Comparación de crecimiento de ingreso total vs autónomo (deciles seleccionados)}
\begin{tabular}{@{}cccc@{}}
\toprule
\textbf{Decil} & \textbf{Crec. Total} & \textbf{Crec. Autón.} & \textbf{Crec. Transf.} \\
\midrule
1 & +51,3\% & +0,4\% & +75,0\% \\
2 & +31,9\% & +15,7\% & +56,7\% \\
3 & +27,4\% & +18,9\% & +44,4\% \\
10 & +18,8\% & +18,9\% & +19,1\% \\
\bottomrule
\end{tabular}
\end{table}

La gran brecha entre el crecimiento total y autónomo en los deciles inferiores revela que las transferencias estatales son el principal impulsor del crecimiento de ingresos para los hogares más pobres. En el decil 1, el ingreso autónomo apenas creció (+0,4\%) mientras el ingreso total creció 51\%, con el ingreso por transferencias creciendo 75\%.

% ============================================================
\section{Análisis de Distribución de Ingresos}
% ============================================================

\subsection{Hogares con Solo Adultos Mayores por Decil de Ingreso}

Usando los límites de deciles de ingreso autónomo, los hogares con solo adultos mayores muestran una divergencia notable entre posiciones de ingreso total y autónomo:

\begin{table}[H]
\centering
\caption{Distribución de hogares con solo adultos mayores por decil de ingreso}
\begin{tabular}{@{}crrrr@{}}
\toprule
\textbf{Decil} & \textbf{Total 2017} & \textbf{Total 2024} & \textbf{Autón. 2017} & \textbf{Autón. 2024} \\
\midrule
1 (más pobre) & 0,28\% & 0,01\% & 18,23\% & 31,71\% \\
2 & 0,22\% & 0,11\% & 8,05\% & 10,66\% \\
3 & 1,05\% & 0,14\% & 9,48\% & 9,21\% \\
\midrule
\textbf{3 inferiores total} & \textbf{1,55\%} & \textbf{0,26\%} & \textbf{35,76\%} & \textbf{51,58\%} \\
\bottomrule
\end{tabular}
\end{table}

\textbf{Observación crítica}:
\begin{itemize}
    \item Cuando se mide por \textbf{ingreso total}, los hogares con solo adultos mayores subieron en la distribución (menos en deciles inferiores)
    \item Cuando se mide por \textbf{ingreso autónomo}, los hogares con solo adultos mayores bajaron---más de la mitad (51,6\%) están ahora en los tres deciles inferiores
\end{itemize}

Esta divergencia revela el rol masivo de las transferencias estatales. Sin transferencias, la posición de ingreso autónomo de los hogares de adultos mayores \textbf{se deterioró significativamente} entre 2017 y 2024. Las transferencias compensan completamente este deterioro y más.

% ============================================================
\section{Ingreso Laboral y Prima Educacional}
% ============================================================

Esta sección analiza el ingreso laboral por hora para trabajadores empleados (dependientes de un empleador, edades 26--65) por cohorte de edad y nivel educacional. Todos los valores de 2017 están ajustados por inflación a pesos chilenos de 2024.

\subsection{Ingreso Laboral por Hora según Cohorte de Edad y Educación}

\begin{table}[H]
\centering
\caption{Ingreso laboral mediano por hora según cohorte de edad y educación (CLP/hora, precios 2024)}
\label{tab:income_median}
\small
\begin{tabular}{@{}lrrrrrr@{}}
\toprule
& \multicolumn{3}{c}{\textbf{2017}} & \multicolumn{3}{c}{\textbf{2024}} \\
\cmidrule(lr){2-4} \cmidrule(lr){5-7}
\textbf{Cohorte} & Total & Sin Univ & Univ & Total & Sin Univ & Univ \\
\midrule
26-30 & 3.156 & 2.536 & 4.339 & 3.750 & 3.125 & 4.545 \\
31-35 & 3.550 & 2.689 & 5.325 & 4.375 & 3.125 & 5.556 \\
36-40 & 3.495 & 2.662 & 5.809 & 4.545 & 3.125 & 6.250 \\
41-46 & 3.156 & 2.662 & 5.917 & 4.545 & 3.125 & 6.818 \\
46-50 & 3.156 & 2.589 & 5.917 & 4.091 & 3.125 & 6.818 \\
51-55 & 3.018 & 2.524 & 6.311 & 3.750 & 3.125 & 6.250 \\
56-60 & 2.966 & 2.485 & 6.311 & 3.571 & 3.125 & 6.667 \\
61-65 & 2.998 & 2.524 & 6.616 & 3.409 & 3.125 & 6.818 \\
\midrule
\textbf{Promedio} & 3.187 & 2.584 & 5.818 & 4.004 & 3.125 & 6.215 \\
\bottomrule
\end{tabular}
\end{table}

\textbf{Observaciones clave}:
\begin{itemize}
    \item El ingreso mediano por hora para trabajadores \textbf{sin educación universitaria} aumentó de 2.584 a 3.125 CLP/hora (+21\%)
    \item El ingreso mediano por hora para trabajadores \textbf{con educación universitaria} aumentó de 5.818 a 6.215 CLP/hora (+7\%)
    \item Los trabajadores sin universidad vieron mayores ganancias proporcionales, estrechando la brecha educacional
\end{itemize}

\subsection{Prima Educacional Universitaria}

La prima educacional mide cuánto más ganan los trabajadores con educación universitaria comparados con aquellos sin educación universitaria.

\begin{table}[H]
\centering
\caption{Prima educacional universitaria sobre ingreso mediano por hora}
\label{tab:premium_median}
\begin{tabular}{@{}lrrr@{}}
\toprule
\textbf{Cohorte} & \textbf{2017} & \textbf{2024} & \textbf{Cambio} \\
\midrule
26-30 & 71,1\% & 45,4\% & --25,7 pp \\
31-35 & 98,0\% & 77,8\% & --20,2 pp \\
36-40 & 118,2\% & 100,0\% & --18,2 pp \\
41-46 & 122,3\% & 118,2\% & --4,1 pp \\
46-50 & 128,5\% & 118,2\% & --10,3 pp \\
51-55 & 150,0\% & 100,0\% & --50,0 pp \\
56-60 & 154,0\% & 113,3\% & --40,7 pp \\
61-65 & 162,1\% & 118,2\% & --43,9 pp \\
\midrule
\textbf{Promedio} & 125,5\% & 98,9\% & \textbf{--26,6 pp} \\
\bottomrule
\end{tabular}
\end{table}

\textbf{Hallazgo notable}: La prima educacional universitaria \textbf{disminuyó sustancialmente} en todas las cohortes de edad. En promedio, la prima mediana cayó de 125,5\% a 98,9\% (una caída de 26,6 puntos porcentuales). Esto significa:
\begin{itemize}
    \item En 2017, los graduados universitarios ganaban 2,26$\times$ el salario por hora de los no graduados
    \item En 2024, los graduados universitarios ganan 1,99$\times$ el salario por hora de los no graduados
\end{itemize}

\subsection{Prima Educacional a través de la Distribución de Ingresos}

Las siguientes figuras muestran cómo la prima educacional cambió en diferentes puntos de la distribución de ingresos por hora.

\begin{figure}[H]
\centering
\includegraphics[width=0.85\textwidth]{../output/figures/education_premium_p25_es.png}
\caption{Prima educacional universitaria en P25 (trabajadores de menores ingresos)}
\label{fig:premium_p25}
\end{figure}

\begin{figure}[H]
\centering
\includegraphics[width=0.85\textwidth]{../output/figures/education_premium_median_es.png}
\caption{Prima educacional universitaria en la mediana}
\label{fig:premium_median}
\end{figure}

\begin{figure}[H]
\centering
\includegraphics[width=0.85\textwidth]{../output/figures/education_premium_p75_es.png}
\caption{Prima educacional universitaria en P75 (trabajadores de mayores ingresos)}
\label{fig:premium_p75}
\end{figure}

\begin{figure}[H]
\centering
\includegraphics[width=0.85\textwidth]{../output/figures/education_premium_p90_es.png}
\caption{Prima educacional universitaria en P90 (trabajadores de ingresos más altos)}
\label{fig:premium_p90}
\end{figure}

\subsection{Resumen: Disminución de Retornos a la Educación Universitaria}

\begin{table}[H]
\centering
\caption{Ingreso por hora y prima educacional por percentil (CLP/hora, precios 2024)}
\label{tab:premium_summary}
\small
\begin{tabular}{@{}lrrrrrrrr@{}}
\toprule
& \multicolumn{3}{c}{\textbf{2017 (adj.)}} & \multicolumn{3}{c}{\textbf{2024}} & \\
\cmidrule(lr){2-4} \cmidrule(lr){5-7}
\textbf{Percentil} & Sin Univ & Univ & Prima & Sin Univ & Univ & Prima & \textbf{Cambio} \\
\midrule
P25 & 2.130 & 3.445 & 61,7\% & 2.784 & 3.921 & 40,8\% & --20,9 pp \\
Mediana & 2.584 & 5.818 & 125,2\% & 3.125 & 6.215 & 98,9\% & --26,3 pp \\
P75 & 3.577 & 10.009 & 179,8\% & 4.028 & 10.306 & 155,9\% & --23,9 pp \\
P90 & 5.219 & 17.054 & 226,8\% & 5.461 & 17.096 & 213,1\% & --13,7 pp \\
\bottomrule
\end{tabular}
\end{table}

\begin{figure}[H]
\centering
\includegraphics[width=0.85\textwidth]{../output/figures/income_by_education_2017_es.png}
\caption{Ingreso laboral por hora según nivel educacional (2017, ajustado por inflación)}
\label{fig:income_by_education_2017}
\end{figure}

\begin{figure}[H]
\centering
\includegraphics[width=0.85\textwidth]{../output/figures/income_by_education_2024_es.png}
\caption{Ingreso laboral por hora según nivel educacional (2024)}
\label{fig:income_by_education_2024}
\end{figure}

La prima educacional disminuyó en todos los puntos de la distribución, con las mayores caídas en la mediana y P75. Esto sugiere:
\begin{itemize}
    \item Los salarios de trabajadores sin universidad crecieron más rápido que los de graduados universitarios
    \item El mercado laboral puede estar experimentando compresión en la mitad y mitad superior de la distribución
    \item Los aumentos del salario mínimo y el ajuste del mercado laboral pueden haber beneficiado desproporcionadamente a trabajadores de menor calificación
\end{itemize}

% ============================================================
\section{Conclusiones Principales}
% ============================================================

\subsection{El Sistema de Transferencias de Chile para Adultos Mayores es Altamente Efectivo}

Las transferencias estatales eliminan \textbf{99,7\%} de la pobreza autónoma entre hogares con solo adultos mayores, reduciendo la pobreza por ingreso total a solo 0,12\%. Esta es una política focalizada notablemente exitosa.

\subsection{Las Transferencias son Centrales para la Reducción de Pobreza}

Las transferencias estatales eliminan aproximadamente 15 puntos porcentuales de pobreza---reduciendo la tasa de pobreza de 25\% (autónoma) a 4,6\% (ingreso total). La eficiencia de las transferencias mejoró de 60,8\% a 76,2\% de la pobreza autónoma eliminada.

\subsection{La Pobreza Autónoma de Adultos Mayores Aumentó, Pero las Transferencias Compensan}

La pobreza autónoma entre hogares con solo adultos mayores \textbf{aumentó} de 31,7\% a 41,7\%. Sin embargo, el sistema de transferencias compensa completamente, reduciendo la pobreza por ingreso total a casi cero. Más de la mitad de los hogares con solo adultos mayores estarían en los tres deciles inferiores de ingreso sin transferencias.

\subsection{Convergencia Regional}

Las regiones históricamente pobres (La Araucanía, Ñuble, Los Lagos) mostraron las mayores reducciones de pobreza, sugiriendo cierta convergencia regional en niveles de vida.

\subsection{Crecimiento Pro-Pobre de Ingresos, Especialmente a Través de Transferencias}

El decil inferior vio un crecimiento de 51\% en ingreso total, pero solo 0,4\% en ingreso autónomo. Esto revela que las transferencias estatales---no las mejoras en ingresos de mercado---impulsaron el crecimiento de ingresos para los hogares más pobres.

\subsection{Disminución de Retornos a la Educación Universitaria}

La prima salarial universitaria cayó de 126\% a 99\% en la mediana, impulsada por un crecimiento salarial más rápido para trabajadores sin universidad (+21\%) comparado con graduados universitarios (+7\%). Esta compresión de la prima educacional ocurrió en todas las cohortes de edad y en todos los puntos de la distribución salarial.

% ============================================================
\section{Fuentes de Datos y Archivos}
% ============================================================

\begin{itemize}
    \item \textbf{Datos Crudos}: \texttt{raw\_data/CASEN\_2017.dta}, \texttt{raw\_data/casen\_2024.dta}
    \item \textbf{Datos Armonizados}: \texttt{harmonized\_data/casen\_subset\_2017.dta}, \texttt{harmonized\_data/casen\_subset\_2024.dta}
    \item \textbf{Análisis de Pobreza}: \texttt{output/poverty\_comparison\_extended\_2017.xlsx}, \texttt{output/poverty\_comparison\_extended\_2024.xlsx}
    \item \textbf{Análisis de Ingreso Laboral}: \texttt{output/labor\_income\_by\_cohort.xlsx}, \texttt{output/labor\_income\_tables.xlsx}
    \item \textbf{Código de Análisis}: \texttt{codes/poverty\_analysis.R}, \texttt{codes/compare\_2017\_2024.R}, \texttt{codes/labor\_income\_by\_cohort.R}
\end{itemize}

% ============================================================
\section{Notas Técnicas}
% ============================================================

\begin{enumerate}
    \item Todas las tasas de pobreza están ponderadas usando factores de expansión de la encuesta (\texttt{expr})
    \item Las líneas de pobreza se basan en el estándar de \$8,3 PPP por día del Banco Mundial (pobreza moderada)
    \item Metodología consistente en ambos años: \texttt{ytotcorh/numper} para ingreso total, \texttt{yautcorh/numper} para ingreso autónomo
    \item El ingreso autónomo (\texttt{yautcorh}) incluye solo ingreso laboral, trabajo independiente y pensiones contributivas---excluye todas las transferencias estatales
    \item Los deciles de ingreso están definidos en base a la distribución del ingreso autónomo
    \item El análisis regional cubre las 16 regiones chilenas
    \item El análisis comunal está limitado a la Región Metropolitana
\end{enumerate}

\vfill

\begin{center}
\rule{0.5\textwidth}{0.4pt}\\[0.5cm]
\textit{Informe generado: Enero 2026}\\
\textit{Datos: CASEN 2017 y CASEN 2024, Ministerio de Desarrollo Social y Familia, Chile}\\[0.3cm]
Código y datos: \href{https://github.com/ceggersp/AnalisisCASEN}{github.com/ceggersp/AnalisisCASEN}
\end{center}

\end{document}
