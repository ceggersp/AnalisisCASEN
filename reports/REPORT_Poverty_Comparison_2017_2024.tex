\documentclass[11pt,a4paper]{article}

% Packages
\usepackage[utf8]{inputenc}
\usepackage[T1]{fontenc}
\usepackage[english]{babel}
\usepackage{geometry}
\usepackage{booktabs}
\usepackage{array}
\usepackage{longtable}
\usepackage{graphicx}
\usepackage{float}
\usepackage{hyperref}
\usepackage{xcolor}
\usepackage{titlesec}
\usepackage{parskip}
\usepackage{fancyhdr}
\usepackage{palatino} % cool font 
\usepackage{mathpazo} % fancy font for math

% Page geometry
\geometry{margin=2.5cm}

% Hyperref setup
\hypersetup{
    colorlinks=true,
    linkcolor=blue!60!black,
    urlcolor=blue!60!black,
    citecolor=blue!60!black
}

% Header/Footer
\pagestyle{fancy}
\fancyhf{}
\rhead{CASEN 2017-2024 Comparison}
\lhead{Income, Poverty, and Inequality in Chile}
\rfoot{Page \thepage}

% Title formatting
\titleformat{\section}{\Large\bfseries}{\thesection}{1em}{}
\titleformat{\subsection}{\large\bfseries}{\thesubsection}{1em}{}

% Custom column type for tables
\newcolumntype{R}[1]{>{\raggedleft\arraybackslash}p{#1}}
\newcolumntype{L}[1]{>{\raggedright\arraybackslash}p{#1}}
\newcolumntype{C}[1]{>{\centering\arraybackslash}p{#1}}

\begin{document}

% Title Page
\begin{titlepage}
    \centering
    \vspace*{2cm}

    {\Huge\bfseries Income, Poverty, and Inequality in Chile}\\[0.5cm]
    {\LARGE A Comparison of CASEN 2017 and 2024}\\[2cm]

    \vfill

    {\large Data Source: CASEN Survey}\\[0.3cm]
    {\large Ministerio de Desarrollo Social y Familia, Chile}\\[1cm]

    {\large January 2026}

    \vfill
\end{titlepage}

% Table of Contents
\tableofcontents
\newpage

% ============================================================
\section{Executive Summary}
% ============================================================

This report analyzes changes in income, poverty, and inequality in Chile between 2017 and 2024, using data from the CASEN (Caracterización Socioeconómica Nacional) surveys. The analysis covers three major dimensions: poverty reduction and state transfer effectiveness, income growth across the distribution, and labor market returns to education.

\textbf{Key Findings}:

\begin{itemize}
    \item \textbf{Poverty reduction}: Total income poverty fell from \textbf{9.8\% to 4.6\%}, with state transfer efficiency nearly doubling (from 27.8\% to 45.4\% of autonomous poverty eliminated). Poverty among elderly-only households is virtually eliminated (0.12\%).

    \item \textbf{Pro-poor income growth}: Income growth was strongly progressive---the bottom decile grew 26\% while the top decile grew only 7\% (inflation-adjusted). This pattern held for both total and autonomous income, indicating market-driven compression.

    \item \textbf{Declining education premium}: The university wage premium fell substantially, from 125\% to 99\% at the median. Hourly wages for non-university workers grew 21\% while university graduates saw only 7\% growth. This represents a significant narrowing of education-based inequality.
\end{itemize}

Together, these findings suggest Chile experienced broad-based income compression between 2017 and 2024, with gains concentrated among lower-income households and less-educated workers.

% ============================================================
\section{Methodology}
% ============================================================

\subsection{Income Definitions}

\begin{itemize}
    \item \textbf{Total income}: \texttt{ytotcorh / numper} (corrected household income per capita, including all sources)
    \item \textbf{Autonomous income}: \texttt{(ytotcorh - ysubh) / numper} (excluding state transfers)
    \item \textbf{State transfers (ysubh)}: Non-contributory pensions + subsidies
    \item \textbf{Contributory pensions (AFP)}: Remain in autonomous income (self-financed)
\end{itemize}

\subsection{Poverty Lines}

\begin{itemize}
    \item \textbf{2017}: \$107,347 per capita per month (2017 prices)
    \item \textbf{2024}: \$152,160 per capita per month (2024 prices)
\end{itemize}

Both poverty lines are equivalent to the World Bank's \$8.3 PPP per day standard, adjusted for Chilean prices in each year.

\subsection{Household Categories}

\begin{table}[H]
\centering
\begin{tabular}{@{}ll@{}}
\toprule
\textbf{Category} & \textbf{Definition} \\
\midrule
No65 & Households with NO members aged 65+ \\
AtLeast1\_65 & Households with at least one member aged 65+ \\
Only65 & Households where ALL members are aged 65+ \\
WorkAge & Households with only members aged 24--64 \\
\bottomrule
\end{tabular}
\caption{Household age composition categories}
\end{table}

% ============================================================
\section{Main Findings}
% ============================================================

\subsection{Dramatic Reduction in Poverty (2017--2024)}

\begin{table}[H]
\centering
\begin{tabular}{@{}lrrr@{}}
\toprule
\textbf{Metric} & \textbf{2017} & \textbf{2024} & \textbf{Change} \\
\midrule
Total income poverty & 9.80\% & 4.59\% & \textbf{--5.21 pp} \\
Autonomous income poverty & 13.57\% & 8.41\% & \textbf{--5.16 pp} \\
Transfer impact & 3.77 pp & 3.82 pp & +0.05 pp \\
\bottomrule
\end{tabular}
\caption{National poverty rates comparison}
\end{table}

Both total and autonomous poverty decreased by similar magnitudes ($\sim$5 pp), suggesting that poverty reduction was driven primarily by improvements in market income (wages, self-employment, contributory pensions) rather than expansion of state transfers.

\subsection{State Transfer Efficiency Doubled}

\begin{table}[H]
\centering
\begin{tabular}{@{}lrr@{}}
\toprule
\textbf{Metric} & \textbf{2017} & \textbf{2024} \\
\midrule
Transfer impact (pp) & 3.77 & 3.82 \\
\% of autonomous poverty eliminated & \textbf{27.8\%} & \textbf{45.4\%} \\
\bottomrule
\end{tabular}
\caption{Transfer efficiency comparison}
\end{table}

Although the absolute transfer impact remained nearly constant (3.77 pp vs 3.82 pp), efficiency increased dramatically because autonomous poverty fell. State transfers now eliminate \textbf{45.4\%} of pre-transfer poverty, compared to only 27.8\% in 2017.

\subsection{Near-Elimination of Poverty Among the Elderly}

The most striking finding concerns \textbf{elderly-only households} (Only65):

\begin{table}[H]
\centering
\begin{tabular}{@{}lrrr@{}}
\toprule
\textbf{Metric} & \textbf{2017} & \textbf{2024} & \textbf{Change} \\
\midrule
Total income poverty & 0.85\% & \textbf{0.12\%} & --0.73 pp \\
Autonomous income poverty & 10.45\% & 9.34\% & --1.11 pp \\
Transfer impact & 9.60 pp & 9.22 pp & --0.38 pp \\
\textbf{Transfer efficiency} & 91.9\% & \textbf{98.7\%} & +6.8 pp \\
\bottomrule
\end{tabular}
\caption{Elderly-only households poverty indicators}
\end{table}

\textbf{Interpretation}:
\begin{itemize}
    \item Without state transfers, $\sim$9--10\% of elderly-only households would be poor (similar in both years)
    \item State transfers now eliminate \textbf{98.7\%} of autonomous poverty among elderly-only households
    \item Total income poverty among elderly-only households has been virtually eliminated (0.12\%)
    \item This represents one of the most effective targeted transfer programs in Latin America
\end{itemize}

\subsection{Poverty Reduction Across All Household Types}

\begin{table}[H]
\centering
\begin{tabular}{@{}lrrr@{}}
\toprule
\textbf{Household Type} & \textbf{2017} & \textbf{2024} & \textbf{Change} \\
\midrule
No elderly (No65) & 11.49\% & 5.71\% & --5.78 pp \\
At least 1 elderly & 5.66\% & 1.73\% & --3.93 pp \\
Elderly-only & 0.85\% & 0.12\% & --0.73 pp \\
Working-age only & 2.78\% & 1.59\% & --1.19 pp \\
\bottomrule
\end{tabular}
\caption{Total income poverty by household type}
\end{table}

All household types experienced poverty reduction, with households without elderly members showing the largest absolute decrease (--5.78 pp).

\subsection{Transfer Efficiency by Household Type}

\begin{table}[H]
\centering
\begin{tabular}{@{}lrrr@{}}
\toprule
\textbf{Household Type} & \textbf{2017} & \textbf{2024} & \textbf{Change} \\
\midrule
Overall & 27.8\% & 45.4\% & +17.6 pp \\
No elderly & 17.6\% & 26.0\% & +8.4 pp \\
At least 1 elderly & 55.3\% & 82.9\% & +27.6 pp \\
\textbf{Elderly-only} & 91.9\% & \textbf{98.7\%} & +6.8 pp \\
Working-age only & 28.5\% & 31.5\% & +3.0 pp \\
\bottomrule
\end{tabular}
\caption{Transfer efficiency (\% of autonomous poverty eliminated) by household type}
\end{table}

\textbf{Key insight}: Households with elderly members benefit most from state transfers. For households with at least one elderly member, transfers now eliminate 82.9\% of autonomous poverty, up from 55.3\% in 2017.

% ============================================================
\section{Regional Analysis}
% ============================================================

\subsection{Regions with Largest Poverty Reduction}

\begin{table}[H]
\centering
\begin{tabular}{@{}lrrr@{}}
\toprule
\textbf{Region} & \textbf{2017} & \textbf{2024} & \textbf{Change} \\
\midrule
La Araucanía & 19.08\% & 9.30\% & \textbf{--9.78 pp} \\
Los Lagos & 12.58\% & 3.60\% & \textbf{--8.98 pp} \\
Ñuble & 15.84\% & 7.10\% & \textbf{--8.74 pp} \\
Coquimbo & 14.89\% & 6.47\% & --8.42 pp \\
Biobío & 13.31\% & 4.99\% & --8.32 pp \\
\bottomrule
\end{tabular}
\caption{Regions with largest poverty reduction}
\end{table}

Historically poor regions in southern Chile (La Araucanía, Los Lagos) and central-south Chile (Ñuble, Maule) showed the largest improvements.

\subsection{Only Region with Poverty Increase}

\begin{table}[H]
\centering
\begin{tabular}{@{}lrrr@{}}
\toprule
\textbf{Region} & \textbf{2017} & \textbf{2024} & \textbf{Change} \\
\midrule
Magallanes & 2.25\% & 2.43\% & +0.18 pp \\
\bottomrule
\end{tabular}
\caption{Region with poverty increase}
\end{table}

Magallanes was the only region where poverty increased slightly, though it remains among the lowest poverty regions.

\subsection{Regions with Largest Increase in Transfer Effectiveness}

\begin{table}[H]
\centering
\begin{tabular}{@{}lrrr@{}}
\toprule
\textbf{Region} & \textbf{Impact 2017} & \textbf{Impact 2024} & \textbf{Change} \\
\midrule
Atacama & 1.81 pp & 3.09 pp & +1.28 pp \\
Biobío & 5.03 pp & 5.66 pp & +0.63 pp \\
Valparaíso & 3.57 pp & 3.95 pp & +0.38 pp \\
\bottomrule
\end{tabular}
\caption{Regions with largest increase in transfer effectiveness}
\end{table}

% ============================================================
\section{Demographic Shifts}
% ============================================================

\subsection{Household Composition Changes (2017--2024)}

\begin{table}[H]
\centering
\begin{tabular}{@{}lrrr@{}}
\toprule
\textbf{Category} & \textbf{2017} & \textbf{2024} & \textbf{Change} \\
\midrule
No elderly & 71.09\% & 71.84\% & +0.75 pp \\
At least 1 elderly & 28.91\% & 28.16\% & --0.75 pp \\
Elderly-only & 5.59\% & 5.77\% & +0.18 pp \\
\textbf{Working-age only} & 13.64\% & \textbf{17.92\%} & \textbf{+4.28 pp} \\
\bottomrule
\end{tabular}
\caption{Population share by household composition}
\end{table}

The share of working-age only households (24--64, no children or elderly) increased by 4.28 percentage points, reflecting demographic changes and possibly delayed childbearing.

% ============================================================
\section{Income Growth by Decile}
% ============================================================

This section analyzes how average per capita income changed across the income distribution between 2017 and 2024. All 2017 values are adjusted for inflation (factor 1.42) to express in 2024 Chilean pesos.

\subsection{Total Income Growth by Decile}

\begin{table}[H]
\centering
\begin{tabular}{@{}crrr@{}}
\toprule
\textbf{Decile} & \textbf{2017 (adj.)} & \textbf{2024} & \textbf{Change} \\
\midrule
1 (poorest) & 111,534 & 141,011 & \textbf{+26.4\%} \\
2 & 172,535 & 221,021 & \textbf{+28.1\%} \\
3 & 215,890 & 275,883 & \textbf{+27.8\%} \\
4 & 258,262 & 330,271 & +27.9\% \\
5 & 304,404 & 386,745 & +27.0\% \\
6 & 359,338 & 451,587 & +25.7\% \\
7 & 431,036 & 532,548 & +23.6\% \\
8 & 535,356 & 649,377 & +21.3\% \\
9 & 732,837 & 862,796 & +17.7\% \\
10 (richest) & 1,813,019 & 1,943,776 & +7.2\% \\
\bottomrule
\end{tabular}
\caption{Average total per capita income by decile (CLP, 2024 prices)}
\end{table}

\textbf{Key finding}: Income growth was \textbf{strongly pro-poor}. The bottom deciles experienced gains of 26--28\%, while the top decile grew only 7.2\%. This represents a significant compression of the income distribution.

\subsection{Autonomous Income Growth by Decile}

\begin{table}[H]
\centering
\begin{tabular}{@{}crrr@{}}
\toprule
\textbf{Decile} & \textbf{2017 (adj.)} & \textbf{2024} & \textbf{Change} \\
\midrule
1 (poorest) & 97,896 & 122,384 & \textbf{+25.0\%} \\
2 & 156,345 & 195,214 & \textbf{+24.9\%} \\
3 & 197,737 & 248,600 & \textbf{+25.7\%} \\
4 & 240,487 & 295,924 & +23.1\% \\
5 & 285,076 & 348,054 & +22.1\% \\
6 & 341,171 & 408,862 & +19.8\% \\
7 & 415,935 & 492,178 & +18.3\% \\
8 & 522,811 & 610,859 & +16.8\% \\
9 & 722,737 & 833,368 & +15.3\% \\
10 (richest) & 1,807,877 & 1,931,629 & +6.8\% \\
\bottomrule
\end{tabular}
\caption{Average autonomous per capita income by decile (CLP, 2024 prices)}
\end{table}

\textbf{Key finding}: Autonomous income (excluding state transfers) also showed pro-poor growth, though slightly less pronounced than total income. This indicates that market income improvements---wages, employment, contributory pensions---benefited lower deciles more than higher deciles.

\subsection{Transfer Contribution by Decile}

The difference between total and autonomous income growth reveals the role of state transfers:

\begin{table}[H]
\centering
\begin{tabular}{@{}ccc@{}}
\toprule
\textbf{Decile} & \textbf{Total Growth} & \textbf{Autonomous Growth} \\
\midrule
1 & +26.4\% & +25.0\% \\
2 & +28.1\% & +24.9\% \\
3 & +27.8\% & +25.7\% \\
10 & +7.2\% & +6.8\% \\
\bottomrule
\end{tabular}
\caption{Comparison of total vs autonomous income growth (selected deciles)}
\end{table}

The small gap between total and autonomous growth (1--3 percentage points in lower deciles) suggests that while transfers contribute to poverty reduction, the bulk of income growth came from market sources.

% ============================================================
\section{Income Distribution Analysis}
% ============================================================

\subsection{Elderly-Only Households by Income Decile}

Using total income decile boundaries, elderly-only households shifted toward higher deciles:

\begin{table}[H]
\centering
\begin{tabular}{@{}crrrr@{}}
\toprule
\textbf{Decile} & \textbf{Total 2017} & \textbf{Total 2024} & \textbf{Auton. 2017} & \textbf{Auton. 2024} \\
\midrule
1 (poorest) & 0.85\% & 0.22\% & 10.51\% & 13.86\% \\
2 & 2.08\% & 1.73\% & 5.39\% & 11.66\% \\
3 & 4.93\% & 2.53\% & 7.05\% & 9.41\% \\
\midrule
\textbf{Bottom 3 total} & \textbf{7.86\%} & \textbf{4.48\%} & \textbf{22.95\%} & \textbf{34.93\%} \\
\bottomrule
\end{tabular}
\caption{Elderly-only household distribution by income decile}
\end{table}

\textbf{Critical observation}:
\begin{itemize}
    \item When measured by \textbf{total income}, elderly-only households moved UP the distribution (fewer in bottom deciles)
    \item When measured by \textbf{autonomous income}, elderly-only households moved DOWN (more in bottom deciles)
\end{itemize}

This divergence reveals that state transfers are doing more work to elevate elderly households out of poverty. Without transfers, 34.9\% of elderly-only households would be in the bottom three deciles (up from 23\% in 2017). Transfers move most of them to middle and upper deciles.

% ============================================================
\section{Labor Income and Education Premium}
% ============================================================

This section analyzes hourly labor income for employed workers (dependent on employer, ages 26--65) by age cohort and education level. All 2017 values are inflation-adjusted to 2024 Chilean pesos.

\subsection{Hourly Labor Income by Age Cohort and Education}

\begin{table}[H]
\centering
\small
\begin{tabular}{@{}lrrrrrr@{}}
\toprule
& \multicolumn{3}{c}{\textbf{2017}} & \multicolumn{3}{c}{\textbf{2024}} \\
\cmidrule(lr){2-4} \cmidrule(lr){5-7}
\textbf{Cohort} & Total & No Univ & Univ & Total & No Univ & Univ \\
\midrule
26-30 & 729 & 586 & 1,002 & 866 & 722 & 1,050 \\
31-35 & 820 & 621 & 1,230 & 1,010 & 722 & 1,283 \\
36-40 & 807 & 615 & 1,342 & 1,050 & 722 & 1,443 \\
41-46 & 729 & 615 & 1,366 & 1,050 & 722 & 1,575 \\
46-50 & 729 & 598 & 1,366 & 945 & 722 & 1,575 \\
51-55 & 697 & 583 & 1,458 & 866 & 722 & 1,443 \\
56-60 & 685 & 574 & 1,458 & 825 & 722 & 1,540 \\
61-65 & 692 & 583 & 1,528 & 787 & 722 & 1,575 \\
\midrule
\textbf{Average} & 736 & 597 & 1,344 & 925 & 722 & 1,436 \\
\bottomrule
\end{tabular}
\caption{Median hourly labor income by age cohort and education (CLP/hour, 2024 prices)}
\label{tab:income_median}
\end{table}

\textbf{Key observations}:
\begin{itemize}
    \item Median hourly income for workers \textbf{without university education} increased from 597 to 722 CLP/hour (+21\%)
    \item Median hourly income for workers \textbf{with university education} increased from 1,344 to 1,436 CLP/hour (+7\%)
    \item Non-university workers saw larger proportional gains, narrowing the education gap
\end{itemize}

\subsection{University Education Premium}

The education premium measures how much more university-educated workers earn compared to those without university education.

\begin{table}[H]
\centering
\begin{tabular}{@{}lrrr@{}}
\toprule
\textbf{Cohort} & \textbf{2017} & \textbf{2024} & \textbf{Change} \\
\midrule
26-30 & 71.0\% & 45.4\% & --25.6 pp \\
31-35 & 98.1\% & 77.7\% & --20.4 pp \\
36-40 & 118.2\% & 99.9\% & --18.3 pp \\
41-46 & 122.1\% & 118.1\% & --4.0 pp \\
46-50 & 128.4\% & 118.1\% & --10.3 pp \\
51-55 & 150.1\% & 99.9\% & --50.2 pp \\
56-60 & 154.0\% & 113.3\% & --40.7 pp \\
61-65 & 162.1\% & 118.1\% & --44.0 pp \\
\midrule
\textbf{Average} & 125.5\% & 98.8\% & \textbf{--26.7 pp} \\
\bottomrule
\end{tabular}
\caption{University education premium on median hourly income}
\label{tab:premium_median}
\end{table}

\textbf{Striking finding}: The university education premium \textbf{declined substantially} across all age cohorts. On average, the median premium fell from 125.5\% to 98.8\% (a drop of 26.7 percentage points). This means:
\begin{itemize}
    \item In 2017, university graduates earned 2.25$\times$ the hourly wage of non-graduates
    \item In 2024, university graduates earn 1.99$\times$ the hourly wage of non-graduates
\end{itemize}

\subsection{Education Premium Across the Income Distribution}

The following figures show how the education premium changed across different points of the hourly income distribution.

\begin{figure}[H]
\centering
\includegraphics[width=0.85\textwidth]{education_premium_p25.png}
\caption{University education premium at P25 (lower-earning workers)}
\label{fig:premium_p25}
\end{figure}

\begin{figure}[H]
\centering
\includegraphics[width=0.85\textwidth]{education_premium_median.png}
\caption{University education premium at median}
\label{fig:premium_median}
\end{figure}

\begin{figure}[H]
\centering
\includegraphics[width=0.85\textwidth]{education_premium_p75.png}
\caption{University education premium at P75 (higher-earning workers)}
\label{fig:premium_p75}
\end{figure}

\begin{figure}[H]
\centering
\includegraphics[width=0.85\textwidth]{education_premium_p90.png}
\caption{University education premium at P90 (top earners)}
\label{fig:premium_p90}
\end{figure}

\subsection{Summary: Declining Returns to University Education}

\begin{table}[H]
\centering
\begin{tabular}{@{}lrrr@{}}
\toprule
\textbf{Percentile} & \textbf{Premium 2017} & \textbf{Premium 2024} & \textbf{Change} \\
\midrule
P25 & 61.7\% & 40.8\% & --20.9 pp \\
Median & 125.5\% & 98.8\% & --26.7 pp \\
P75 & 179.6\% & 155.4\% & --24.2 pp \\
P90 & 226.4\% & 211.8\% & --14.6 pp \\
\bottomrule
\end{tabular}
\caption{Average education premium by percentile (all cohorts)}
\label{tab:premium_summary}
\end{table}

The education premium declined at all points of the distribution, with the largest decreases at the median and P75. This suggests:
\begin{itemize}
    \item Wages for non-university workers grew faster than for university graduates
    \item The labor market may be experiencing compression at the middle and upper-middle of the distribution
    \item Minimum wage increases and labor market tightening may have disproportionately benefited lower-skilled workers
\end{itemize}

% ============================================================
\section{Key Takeaways}
% ============================================================

\subsection{Chile's Transfer System for the Elderly is Highly Effective}

State transfers eliminate \textbf{98.7\%} of autonomous poverty among elderly-only households, bringing total income poverty down to just 0.12\%. This is a remarkably successful targeted policy.

\subsection{Overall Poverty Reduction Driven by Market Income Improvements}

The parallel decrease in both total and autonomous poverty (--5.2 pp each) indicates that economic growth, employment, and market wages improved substantially between 2017 and 2024.

\subsection{Transfer Efficiency Nearly Doubled}

Even though the absolute poverty reduction from transfers remained constant ($\sim$3.8 pp), transfers now eliminate a much larger share of autonomous poverty (45\% vs 28\%), as the base of autonomous poor decreased.

\subsection{Regional Convergence}

Historically poor regions (La Araucanía, Ñuble, Los Lagos) showed the largest poverty reductions, suggesting some regional convergence in living standards.

\subsection{Growing Dependence of Elderly on Transfers}

Without transfers, more elderly households would fall into poverty in 2024 than in 2017 (as measured by autonomous income distribution). The transfer system is compensating for what would otherwise be a deteriorating position of elderly households in the income distribution.

\subsection{Declining Returns to University Education}

The university wage premium fell from 125\% to 99\% at the median, driven by faster wage growth for non-university workers (+21\%) compared to university graduates (+7\%). This compression of the education premium occurred across all age cohorts and at all points of the wage distribution, suggesting structural changes in the labor market that favor less-educated workers.

% ============================================================
\section{Data Sources and Files}
% ============================================================

\begin{itemize}
    \item \textbf{Raw Data}: \texttt{raw\_data/CASEN\_2017.dta}, \texttt{raw\_data/casen\_2024.dta}
    \item \textbf{Harmonized Data}: \texttt{harmonized\_data/casen\_subset\_2017.dta}, \texttt{harmonized\_data/casen\_subset\_2024.dta}
    \item \textbf{Poverty Analysis}: \texttt{output/poverty\_comparison\_extended\_2017.xlsx}, \texttt{output/poverty\_comparison\_extended\_2024.xlsx}
    \item \textbf{Labor Income Analysis}: \texttt{output/labor\_income\_by\_cohort.xlsx}, \texttt{output/labor\_income\_tables.xlsx}
    \item \textbf{Analysis Code}: \texttt{codes/poverty\_analysis.R}, \texttt{codes/compare\_2017\_2024.R}, \texttt{codes/labor\_income\_by\_cohort.R}
\end{itemize}

% ============================================================
\section{Technical Notes}
% ============================================================

\begin{enumerate}
    \item All poverty rates are weighted using survey expansion factors (\texttt{expr})
    \item Poverty lines are based on World Bank's \$8.3 PPP per day (moderate poverty)
    \item Consistent methodology across both years: \texttt{ytotcorh/numper} for total income, \texttt{(ytotcorh-ysubh)/numper} for autonomous income
    \item Regional analysis covers all 16 Chilean regions
    \item Comunal analysis limited to Región Metropolitana
\end{enumerate}

\vfill

\begin{center}
\rule{0.5\textwidth}{0.4pt}\\[0.5cm]
\textit{Report generated: January 2026}\\
\textit{Data: CASEN 2017 and CASEN 2024, Ministerio de Desarrollo Social y Familia, Chile}
\end{center}

\end{document}
