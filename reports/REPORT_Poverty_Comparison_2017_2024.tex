\documentclass[11pt,a4paper]{article}

% Packages
\usepackage[utf8]{inputenc}
\usepackage[T1]{fontenc}
\usepackage[english]{babel}
\usepackage{geometry}
\usepackage{booktabs}
\usepackage{array}
\usepackage{longtable}
\usepackage{graphicx}
\usepackage{float}
\usepackage{hyperref}
\usepackage{xcolor}
\usepackage{titlesec}
\usepackage{parskip}
\usepackage{fancyhdr}
\usepackage{palatino} % cool font 
\usepackage{mathpazo} % fancy font for math

% Page geometry
\geometry{margin=2.5cm}

% Hyperref setup
\hypersetup{
    colorlinks=true,
    linkcolor=blue!60!black,
    urlcolor=blue!60!black,
    citecolor=blue!60!black
}

% Header/Footer
\pagestyle{fancy}
\fancyhf{}
\rhead{CASEN 2017-2024 Comparison}
\lhead{Income, Poverty, and Inequality in Chile}
\rfoot{Page \thepage}

% Title formatting
\titleformat{\section}{\Large\bfseries}{\thesection}{1em}{}
\titleformat{\subsection}{\large\bfseries}{\thesubsection}{1em}{}

% Custom column type for tables
\newcolumntype{R}[1]{>{\raggedleft\arraybackslash}p{#1}}
\newcolumntype{L}[1]{>{\raggedright\arraybackslash}p{#1}}
\newcolumntype{C}[1]{>{\centering\arraybackslash}p{#1}}

\begin{document}

% Title Page
\begin{titlepage}
    \centering
    \vspace*{2cm}

    {\Huge\bfseries Income, Poverty, and Inequality in Chile}\\[0.5cm]
    {\LARGE A Comparison of CASEN 2017 and 2024}\\[2cm]

    \vfill

    {\large Data Source: CASEN Survey}\\[0.3cm]
    {\large Ministerio de Desarrollo Social y Familia, Chile}\\[1cm]

    {\large January 2026}

    \vfill
\end{titlepage}

% Table of Contents
\tableofcontents
\newpage

% ============================================================
\section{Executive Summary}
% ============================================================

This report analyzes changes in income, poverty, and inequality in Chile between 2017 and 2024, using data from the CASEN (Caracterización Socioeconómica Nacional) surveys. The analysis covers three major dimensions: poverty reduction and state transfer effectiveness, income growth across the distribution, and labor market returns to education.

\textbf{Key Findings}:

\begin{itemize}
    \item \textbf{Poverty reduction}: Total income poverty fell from \textbf{9.8\% to 4.6\%}, while autonomous income poverty fell from 25.0\% to 19.3\%. State transfer efficiency improved substantially (from 60.8\% to 76.2\% of autonomous poverty eliminated). Poverty among elderly-only households is virtually eliminated (0.12\%).

    \item \textbf{Massive transfer impact on elderly}: Autonomous poverty among elderly-only households increased from 31.7\% to 41.7\%, yet total income poverty fell to near zero (0.12\%). State transfers now eliminate 99.7\% of autonomous poverty among elderly-only households.

    \item \textbf{Pro-poor income growth}: Income growth was strongly progressive---the bottom decile grew 51\% in total income while the top decile grew only 19\% (inflation-adjusted). Autonomous income also showed pro-poor patterns.

    \item \textbf{Declining education premium}: The university wage premium fell substantially, from 126\% to 99\% at the median. Hourly wages for non-university workers grew 21\% while university graduates saw only 7\% growth.
\end{itemize}

Together, these findings suggest Chile experienced broad-based income compression between 2017 and 2024, driven by both expanding state transfers and market income improvements concentrated among lower-income households.

% ============================================================
\section{Methodology}
% ============================================================

\subsection{Income Definitions}

\begin{itemize}
    \item \textbf{Total income}: \texttt{ytotcorh / numper} (corrected household income per capita, including all sources)
    \item \textbf{Autonomous income}: \texttt{yautcorh / numper} (includes only labor income, self-employment, and contributory pensions---excludes all state transfers)
    \item \textbf{State transfers}: Non-contributory pensions (PGU, PBS) + subsidies + other government transfers
    \item \textbf{Contributory pensions (AFP)}: Included in autonomous income (self-financed)
\end{itemize}

\subsection{Poverty Lines}

\begin{itemize}
    \item \textbf{2017}: \$107,347 per capita per month (2017 prices)
    \item \textbf{2024}: \$152,160 per capita per month (2024 prices)
\end{itemize}

Both poverty lines are equivalent to the World Bank's \$8.3 PPP per day standard, adjusted for Chilean prices in each year.

\subsection{Household Categories}

\begin{table}[H]
\centering
\caption{Household age composition categories}
\begin{tabular}{@{}ll@{}}
\toprule
\textbf{Category} & \textbf{Definition} \\
\midrule
No65 & Households with NO members aged 65+ \\
AtLeast1\_65 & Households with at least one member aged 65+ \\
Only65 & Households where ALL members are aged 65+ \\
WorkAge & Households with only members aged 24--64 \\
\bottomrule
\end{tabular}
\end{table}

% ============================================================
\section{Main Findings}
% ============================================================

\subsection{Dramatic Reduction in Poverty (2017--2024)}

\begin{table}[H]
\centering
\caption{National poverty rates comparison}
\begin{tabular}{@{}lrrr@{}}
\toprule
\textbf{Metric} & \textbf{2017} & \textbf{2024} & \textbf{Change} \\
\midrule
Total income poverty & 9.80\% & 4.59\% & \textbf{--5.21 pp} \\
Autonomous income poverty & 25.01\% & 19.32\% & \textbf{--5.69 pp} \\
Transfer impact & 15.21 pp & 14.73 pp & --0.48 pp \\
\bottomrule
\end{tabular}
\end{table}

State transfers have a massive impact on poverty reduction, eliminating approximately 15 percentage points of poverty in both years. Both total and autonomous poverty decreased by similar magnitudes ($\sim$5 pp), indicating that market income improvements and transfer effectiveness both contributed to poverty reduction.

\subsection{State Transfer Efficiency Improved}

\begin{table}[H]
\centering
\caption{Transfer efficiency comparison}
\begin{tabular}{@{}lrr@{}}
\toprule
\textbf{Metric} & \textbf{2017} & \textbf{2024} \\
\midrule
Transfer impact (pp) & 15.21 & 14.73 \\
\% of autonomous poverty eliminated & \textbf{60.8\%} & \textbf{76.2\%} \\
\bottomrule
\end{tabular}
\end{table}

State transfers eliminate a substantial share of autonomous poverty. The efficiency improved from 60.8\% to 76.2\%, meaning transfers now eliminate over three-quarters of pre-transfer poverty. This improvement occurred even as the absolute impact remained roughly constant ($\sim$15 pp).

\subsection{Near-Elimination of Poverty Among the Elderly}

The most striking finding concerns \textbf{elderly-only households} (Only65):

\begin{table}[H]
\centering
\caption{Elderly-only households poverty indicators}
\begin{tabular}{@{}lrrr@{}}
\toprule
\textbf{Metric} & \textbf{2017} & \textbf{2024} & \textbf{Change} \\
\midrule
Total income poverty & 0.85\% & \textbf{0.12\%} & --0.73 pp \\
Autonomous income poverty & 31.71\% & 41.72\% & +10.01 pp \\
Transfer impact & 30.86 pp & 41.60 pp & +10.74 pp \\
\textbf{Transfer efficiency} & 97.3\% & \textbf{99.7\%} & +2.4 pp \\
\bottomrule
\end{tabular}
\end{table}

\textbf{Interpretation}:
\begin{itemize}
    \item Without state transfers, autonomous poverty among elderly-only households \textbf{increased} from 31.7\% to 41.7\%
    \item However, state transfers now eliminate \textbf{99.7\%} of autonomous poverty among elderly-only households
    \item Total income poverty has been virtually eliminated (0.12\%)
    \item This represents one of the most effective targeted transfer programs in Latin America, compensating for deteriorating autonomous income positions
\end{itemize}

\subsection{Poverty Reduction Across All Household Types}

\begin{table}[H]
\centering
\caption{Total income poverty by household type}
\begin{tabular}{@{}lrrr@{}}
\toprule
\textbf{Household Type} & \textbf{2017} & \textbf{2024} & \textbf{Change} \\
\midrule
No elderly (No65) & 11.49\% & 5.71\% & --5.78 pp \\
At least 1 elderly & 5.66\% & 1.73\% & --3.93 pp \\
Elderly-only & 0.85\% & 0.12\% & --0.73 pp \\
Working-age only & 2.78\% & 1.59\% & --1.19 pp \\
\bottomrule
\end{tabular}
\end{table}

All household types experienced poverty reduction, with households without elderly members showing the largest absolute decrease (--5.78 pp).

\subsection{Transfer Efficiency by Household Type}

\begin{table}[H]
\centering
\caption{Transfer efficiency (\% of autonomous poverty eliminated) by household type}
\begin{tabular}{@{}lrrr@{}}
\toprule
\textbf{Household Type} & \textbf{2017} & \textbf{2024} & \textbf{Change} \\
\midrule
Overall & 60.8\% & 76.2\% & +15.4 pp \\
No elderly & 52.4\% & 64.2\% & +11.8 pp \\
At least 1 elderly & 79.2\% & 93.8\% & +14.6 pp \\
\textbf{Elderly-only} & 97.3\% & \textbf{99.7\%} & +2.4 pp \\
Working-age only & 70.6\% & 79.7\% & +9.1 pp \\
\bottomrule
\end{tabular}
\end{table}

\textbf{Key insight}: Transfer efficiency improved across all household types. For households with at least one elderly member, transfers now eliminate 93.8\% of autonomous poverty, up from 79.2\% in 2017. For elderly-only households, efficiency reaches 99.7\%.

% ============================================================
\section{Regional Analysis}
% ============================================================

\subsection{Regions with Largest Poverty Reduction}

\begin{table}[H]
\centering
\caption{Regions with largest poverty reduction}
\begin{tabular}{@{}lrrr@{}}
\toprule
\textbf{Region} & \textbf{2017} & \textbf{2024} & \textbf{Change} \\
\midrule
La Araucanía & 19.08\% & 9.30\% & \textbf{--9.78 pp} \\
Los Lagos & 12.58\% & 3.60\% & \textbf{--8.98 pp} \\
Ñuble & 15.84\% & 7.10\% & \textbf{--8.74 pp} \\
Coquimbo & 14.89\% & 6.47\% & --8.42 pp \\
Biobío & 13.31\% & 4.99\% & --8.32 pp \\
\bottomrule
\end{tabular}
\end{table}

Historically poor regions in southern Chile (La Araucanía, Los Lagos) and central-south Chile (Ñuble, Maule) showed the largest improvements.

\subsection{Only Region with Poverty Increase}

\begin{table}[H]
\centering
\caption{Region with poverty increase}
\begin{tabular}{@{}lrrr@{}}
\toprule
\textbf{Region} & \textbf{2017} & \textbf{2024} & \textbf{Change} \\
\midrule
Magallanes & 2.25\% & 2.43\% & +0.18 pp \\
\bottomrule
\end{tabular}
\end{table}

Magallanes was the only region where poverty increased slightly, though it remains among the lowest poverty regions.

\subsection{Regions with Largest Increase in Transfer Effectiveness}

\begin{table}[H]
\centering
\caption{Regions with largest increase in transfer effectiveness}
\begin{tabular}{@{}lrrr@{}}
\toprule
\textbf{Region} & \textbf{Impact 2017} & \textbf{Impact 2024} & \textbf{Change} \\
\midrule
Atacama & 1.81 pp & 3.09 pp & +1.28 pp \\
Biobío & 5.03 pp & 5.66 pp & +0.63 pp \\
Valparaíso & 3.57 pp & 3.95 pp & +0.38 pp \\
\bottomrule
\end{tabular}
\end{table}

% ============================================================
\section{Demographic Shifts}
% ============================================================

\subsection{Household Composition Changes (2017--2024)}

\begin{table}[H]
\centering
\caption{Population share by household composition}
\begin{tabular}{@{}lrrr@{}}
\toprule
\textbf{Category} & \textbf{2017} & \textbf{2024} & \textbf{Change} \\
\midrule
No elderly & 71.09\% & 71.84\% & +0.75 pp \\
At least 1 elderly & 28.91\% & 28.16\% & --0.75 pp \\
Elderly-only & 5.59\% & 5.77\% & +0.18 pp \\
\textbf{Working-age only} & 13.64\% & \textbf{17.92\%} & \textbf{+4.28 pp} \\
\bottomrule
\end{tabular}
\end{table}

The share of working-age only households (24--64, no children or elderly) increased by 4.28 percentage points, reflecting demographic changes and possibly delayed childbearing.

% ============================================================
\section{Income Growth by Decile}
% ============================================================

This section analyzes how average per capita income changed across the income distribution between 2017 and 2024. All 2017 values are adjusted for inflation (factor 1.42) to express in 2024 Chilean pesos.

\subsection{Total Income Growth by Decile}

Note: Deciles are defined based on \textbf{autonomous income} distribution.

\begin{table}[H]
\centering
\caption{Average total per capita income by decile (CLP, 2024 prices)}
\begin{tabular}{@{}crrr@{}}
\toprule
\textbf{Decile} & \textbf{2017 (adj.)} & \textbf{2024} & \textbf{Change} \\
\midrule
1 (poorest) & 157,222 & 237,807 & \textbf{+51.3\%} \\
2 & 187,385 & 247,117 & \textbf{+31.9\%} \\
3 & 230,425 & 293,611 & \textbf{+27.4\%} \\
4 & 274,886 & 344,358 & +25.3\% \\
5 & 315,314 & 401,367 & +27.3\% \\
6 & 374,961 & 471,540 & +25.8\% \\
7 & 446,590 & 571,221 & +27.9\% \\
8 & 555,216 & 715,263 & +28.8\% \\
9 & 768,212 & 1,006,468 & +31.0\% \\
10 (richest) & 1,899,884 & 2,256,871 & +18.8\% \\
\bottomrule
\end{tabular}
\end{table}

\textbf{Key finding}: Income growth was \textbf{strongly pro-poor}. The bottom decile experienced gains of 51\%, driven largely by state transfers, while the top decile grew only 19\%. This represents a significant compression of the income distribution.

\subsection{Autonomous Income Growth by Decile}

\begin{table}[H]
\centering
\caption{Average autonomous per capita income by decile (CLP, 2024 prices)}
\begin{tabular}{@{}crrr@{}}
\toprule
\textbf{Decile} & \textbf{2017 (adj.)} & \textbf{2024} & \textbf{Change} \\
\midrule
1 (poorest) & 50,063 & 50,259 & \textbf{+0.4\%} \\
2 & 113,552 & 131,421 & +15.7\% \\
3 & 153,330 & 182,307 & +18.9\% \\
4 & 193,783 & 234,684 & +21.1\% \\
5 & 237,185 & 290,200 & +22.4\% \\
6 & 289,602 & 360,751 & +24.6\% \\
7 & 359,415 & 454,437 & +26.4\% \\
8 & 461,249 & 590,804 & +28.1\% \\
9 & 657,777 & 861,967 & +31.0\% \\
10 (richest) & 1,713,050 & 2,037,644 & +18.9\% \\
\bottomrule
\end{tabular}
\end{table}

\textbf{Key finding}: Autonomous income growth was relatively flat at the bottom (decile 1 grew only 0.4\%) but stronger in the middle and upper-middle deciles. The combination of flat autonomous income growth at the bottom with strong total income growth (51\%) reveals the crucial role of state transfers in lifting the poorest households.

\subsection{Transfer Contribution by Decile}

The difference between total and autonomous income growth reveals the critical role of state transfers:

\begin{table}[H]
\centering
\caption{Comparison of total vs autonomous income growth (selected deciles)}
\begin{tabular}{@{}cccc@{}}
\toprule
\textbf{Decile} & \textbf{Total Growth} & \textbf{Auton. Growth} & \textbf{Transfer Growth} \\
\midrule
1 & +51.3\% & +0.4\% & +75.0\% \\
2 & +31.9\% & +15.7\% & +56.7\% \\
3 & +27.4\% & +18.9\% & +44.4\% \\
10 & +18.8\% & +18.9\% & +19.1\% \\
\bottomrule
\end{tabular}
\end{table}

The large gap between total and autonomous growth in the bottom deciles reveals that state transfers are the primary driver of income growth for the poorest households. In decile 1, autonomous income barely grew (+0.4\%) while total income grew 51\%, with transfer income growing 75\%.

% ============================================================
\section{Income Distribution Analysis}
% ============================================================

\subsection{Elderly-Only Households by Income Decile}

Using autonomous income decile boundaries, elderly-only households show a striking divergence between total and autonomous income positions:

\begin{table}[H]
\centering
\caption{Elderly-only household distribution by income decile}
\begin{tabular}{@{}crrrr@{}}
\toprule
\textbf{Decile} & \textbf{Total 2017} & \textbf{Total 2024} & \textbf{Auton. 2017} & \textbf{Auton. 2024} \\
\midrule
1 (poorest) & 0.28\% & 0.01\% & 18.23\% & 31.71\% \\
2 & 0.22\% & 0.11\% & 8.05\% & 10.66\% \\
3 & 1.05\% & 0.14\% & 9.48\% & 9.21\% \\
\midrule
\textbf{Bottom 3 total} & \textbf{1.55\%} & \textbf{0.26\%} & \textbf{35.76\%} & \textbf{51.58\%} \\
\bottomrule
\end{tabular}
\end{table}

\textbf{Critical observation}:
\begin{itemize}
    \item When measured by \textbf{total income}, elderly-only households moved UP the distribution (fewer in bottom deciles)
    \item When measured by \textbf{autonomous income}, elderly-only households moved DOWN---over half (51.6\%) are now in the bottom three deciles
\end{itemize}

This divergence reveals the massive role of state transfers. Without transfers, the autonomous income position of elderly households \textbf{deteriorated significantly} between 2017 and 2024. Transfers completely compensate for this deterioration and more.

% ============================================================
\section{Labor Income and Education Premium}
% ============================================================

This section analyzes hourly labor income for employed workers (dependent on employer, ages 26--65) by age cohort and education level. All 2017 values are inflation-adjusted to 2024 Chilean pesos.

\subsection{Hourly Labor Income by Age Cohort and Education}

\begin{table}[H]
\centering
\caption{Median hourly labor income by age cohort and education (CLP/hour, 2024 prices)}
\label{tab:income_median}
\small
\begin{tabular}{@{}lrrrrrr@{}}
\toprule
& \multicolumn{3}{c}{\textbf{2017}} & \multicolumn{3}{c}{\textbf{2024}} \\
\cmidrule(lr){2-4} \cmidrule(lr){5-7}
\textbf{Cohort} & Total & No Univ & Univ & Total & No Univ & Univ \\
\midrule
26-30 & 3,156 & 2,536 & 4,339 & 3,750 & 3,125 & 4,545 \\
31-35 & 3,550 & 2,689 & 5,325 & 4,375 & 3,125 & 5,556 \\
36-40 & 3,495 & 2,662 & 5,809 & 4,545 & 3,125 & 6,250 \\
41-46 & 3,156 & 2,662 & 5,917 & 4,545 & 3,125 & 6,818 \\
46-50 & 3,156 & 2,589 & 5,917 & 4,091 & 3,125 & 6,818 \\
51-55 & 3,018 & 2,524 & 6,311 & 3,750 & 3,125 & 6,250 \\
56-60 & 2,966 & 2,485 & 6,311 & 3,571 & 3,125 & 6,667 \\
61-65 & 2,998 & 2,524 & 6,616 & 3,409 & 3,125 & 6,818 \\
\midrule
\textbf{Average} & 3,187 & 2,584 & 5,818 & 4,004 & 3,125 & 6,215 \\
\bottomrule
\end{tabular}
\end{table}

\textbf{Key observations}:
\begin{itemize}
    \item Median hourly income for workers \textbf{without university education} increased from 2,584 to 3,125 CLP/hour (+21\%)
    \item Median hourly income for workers \textbf{with university education} increased from 5,818 to 6,215 CLP/hour (+7\%)
    \item Non-university workers saw larger proportional gains, narrowing the education gap
\end{itemize}

\subsection{University Education Premium}

The education premium measures how much more university-educated workers earn compared to those without university education.

\begin{table}[H]
\centering
\caption{University education premium on median hourly income}
\label{tab:premium_median}
\begin{tabular}{@{}lrrr@{}}
\toprule
\textbf{Cohort} & \textbf{2017} & \textbf{2024} & \textbf{Change} \\
\midrule
26-30 & 71.1\% & 45.4\% & --25.7 pp \\
31-35 & 98.0\% & 77.8\% & --20.2 pp \\
36-40 & 118.2\% & 100.0\% & --18.2 pp \\
41-46 & 122.3\% & 118.2\% & --4.1 pp \\
46-50 & 128.5\% & 118.2\% & --10.3 pp \\
51-55 & 150.0\% & 100.0\% & --50.0 pp \\
56-60 & 154.0\% & 113.3\% & --40.7 pp \\
61-65 & 162.1\% & 118.2\% & --43.9 pp \\
\midrule
\textbf{Average} & 125.5\% & 98.9\% & \textbf{--26.6 pp} \\
\bottomrule
\end{tabular}
\end{table}

\textbf{Striking finding}: The university education premium \textbf{declined substantially} across all age cohorts. On average, the median premium fell from 125.5\% to 98.9\% (a drop of 26.6 percentage points). This means:
\begin{itemize}
    \item In 2017, university graduates earned 2.26$\times$ the hourly wage of non-graduates
    \item In 2024, university graduates earn 1.99$\times$ the hourly wage of non-graduates
\end{itemize}

\subsection{Education Premium Across the Income Distribution}

The following figures show how the education premium changed across different points of the hourly income distribution.

\begin{figure}[H]
\centering
\includegraphics[width=0.85\textwidth]{../output/figures/education_premium_p25.png}
\caption{University education premium at P25 (lower-earning workers)}
\label{fig:premium_p25}
\end{figure}

\begin{figure}[H]
\centering
\includegraphics[width=0.85\textwidth]{../output/figures/education_premium_median.png}
\caption{University education premium at median}
\label{fig:premium_median}
\end{figure}

\begin{figure}[H]
\centering
\includegraphics[width=0.85\textwidth]{../output/figures/education_premium_p75.png}
\caption{University education premium at P75 (higher-earning workers)}
\label{fig:premium_p75}
\end{figure}

\begin{figure}[H]
\centering
\includegraphics[width=0.85\textwidth]{../output/figures/education_premium_p90.png}
\caption{University education premium at P90 (top earners)}
\label{fig:premium_p90}
\end{figure}

\subsection{Summary: Declining Returns to University Education}

\begin{table}[H]
\centering
\caption{Hourly income and education premium by percentile (CLP/hour, 2024 prices)}
\label{tab:premium_summary}
\small
\begin{tabular}{@{}lrrrrrrrr@{}}
\toprule
& \multicolumn{3}{c}{\textbf{2017 (adj.)}} & \multicolumn{3}{c}{\textbf{2024}} & \\
\cmidrule(lr){2-4} \cmidrule(lr){5-7}
\textbf{Percentile} & No Univ & Univ & Premium & No Univ & Univ & Premium & \textbf{Change} \\
\midrule
P25 & 2,130 & 3,445 & 61.7\% & 2,784 & 3,921 & 40.8\% & --20.9 pp \\
Median & 2,584 & 5,818 & 125.2\% & 3,125 & 6,215 & 98.9\% & --26.3 pp \\
P75 & 3,577 & 10,009 & 179.8\% & 4,028 & 10,306 & 155.9\% & --23.9 pp \\
P90 & 5,219 & 17,054 & 226.8\% & 5,461 & 17,096 & 213.1\% & --13.7 pp \\
\bottomrule
\end{tabular}
\end{table}

\begin{figure}[H]
\centering
\includegraphics[width=0.85\textwidth]{../output/figures/income_by_education_2017.png}
\caption{Hourly labor income by education level (2017, inflation-adjusted)}
\label{fig:income_by_education_2017}
\end{figure}

\begin{figure}[H]
\centering
\includegraphics[width=0.85\textwidth]{../output/figures/income_by_education_2024.png}
\caption{Hourly labor income by education level (2024)}
\label{fig:income_by_education_2024}
\end{figure}

The education premium declined at all points of the distribution, with the largest decreases at the median and P75. This suggests:
\begin{itemize}
    \item Wages for non-university workers grew faster than for university graduates
    \item The labor market may be experiencing compression at the middle and upper-middle of the distribution
    \item Minimum wage increases and labor market tightening may have disproportionately benefited lower-skilled workers
\end{itemize}

% ============================================================
\section{Key Takeaways}
% ============================================================

\subsection{Chile's Transfer System for the Elderly is Highly Effective}

State transfers eliminate \textbf{99.7\%} of autonomous poverty among elderly-only households, bringing total income poverty down to just 0.12\%. This is a remarkably successful targeted policy.

\subsection{Transfers are Central to Poverty Reduction}

State transfers eliminate approximately 15 percentage points of poverty---bringing the poverty rate from 25\% (autonomous) to 4.6\% (total income). Transfer efficiency improved from 60.8\% to 76.2\% of autonomous poverty eliminated.

\subsection{Autonomous Poverty Among Elderly Increased, But Transfers Compensate}

Autonomous poverty among elderly-only households \textbf{increased} from 31.7\% to 41.7\%. However, the transfer system completely compensates, reducing total income poverty to near zero. Over half of elderly-only households would be in the bottom three income deciles without transfers.

\subsection{Regional Convergence}

Historically poor regions (La Araucanía, Ñuble, Los Lagos) showed the largest poverty reductions, suggesting some regional convergence in living standards.

\subsection{Pro-Poor Income Growth, Especially Through Transfers}

The bottom decile saw 51\% growth in total income, but only 0.4\% in autonomous income. This reveals that state transfers---not market income improvements---drove income growth for the poorest households.

\subsection{Declining Returns to University Education}

The university wage premium fell from 126\% to 99\% at the median, driven by faster wage growth for non-university workers (+21\%) compared to university graduates (+7\%). This compression of the education premium occurred across all age cohorts and at all points of the wage distribution.

% ============================================================
\section{Data Sources and Files}
% ============================================================

\begin{itemize}
    \item \textbf{Raw Data}: \texttt{raw\_data/CASEN\_2017.dta}, \texttt{raw\_data/casen\_2024.dta}
    \item \textbf{Harmonized Data}: \texttt{harmonized\_data/casen\_subset\_2017.dta}, \texttt{harmonized\_data/casen\_subset\_2024.dta}
    \item \textbf{Poverty Analysis}: \texttt{output/poverty\_comparison\_extended\_2017.xlsx}, \texttt{output/poverty\_comparison\_extended\_2024.xlsx}
    \item \textbf{Labor Income Analysis}: \texttt{output/labor\_income\_by\_cohort.xlsx}, \texttt{output/labor\_income\_tables.xlsx}
    \item \textbf{Analysis Code}: \texttt{codes/poverty\_analysis.R}, \texttt{codes/compare\_2017\_2024.R}, \texttt{codes/labor\_income\_by\_cohort.R}
\end{itemize}

% ============================================================
\section{Technical Notes}
% ============================================================

\begin{enumerate}
    \item All poverty rates are weighted using survey expansion factors (\texttt{expr})
    \item Poverty lines are based on World Bank's \$8.3 PPP per day (moderate poverty)
    \item Consistent methodology across both years: \texttt{ytotcorh/numper} for total income, \texttt{yautcorh/numper} for autonomous income
    \item Autonomous income (\texttt{yautcorh}) includes only labor income, self-employment, and contributory pensions---excludes all state transfers
    \item Income deciles are defined based on autonomous income distribution
    \item Regional analysis covers all 16 Chilean regions
    \item Comunal analysis limited to Región Metropolitana
\end{enumerate}

\vfill

\begin{center}
\rule{0.5\textwidth}{0.4pt}\\[0.5cm]
\textit{Report generated: January 2026}\\
\textit{Data: CASEN 2017 and CASEN 2024, Ministerio de Desarrollo Social y Familia, Chile}\\[0.3cm]
Code and data: \href{https://github.com/ceggersp/AnalisisCASEN}{github.com/ceggersp/AnalisisCASEN}
\end{center}

\end{document}
